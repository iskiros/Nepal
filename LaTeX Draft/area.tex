
\section{Study Area}
% Provide a description of the sampling methods, field area, and geology.
% - Include maps or figures if necessary.

% want table with catchment characteristics
% also remember to include a table with abbreviations

\begin{itemize}

\item{What is the lat long extent of the study area?}

85.441 - 85.601 E, 27.822 - 28.157 N

\item{What is the elevation range of the catchment?}

\item{What is the area that the river drains? How do you calculate that (i.e. DEM)?}

\item{What is the geology of the area?}

\item{What is the climate of the area? what is it influenced by? i.e. monsoon, etc.}

Bookhagen and Burbank (2010) identify two main climatic influences in the Himalayas: the monsoon system and the westerlies. 
The westerly winds, typical of this latitude, are responsible for the dry season in the Himalayas.
The monsoon system is further divided into the East Asian and Indian Monsoon systems, which interact with each other.

The high elevation of the High Himalayas creates a barrier that affects atmospheric circulation. 
Bookhagen et al. (2005b) suggest that the Tibetan Plateau's high elevation generates a low-pressure cell near the surface, 
altering atmospheric circulation patterns. 
This is one explanation for the monsoon, though other studies present differing views (see Bookhagen and Burbank, 2010 for a discussion).

The source of precipitation during the Indian Summer Monsoon (ISM) affecting Melamchi is the Bay of Bengal, 
due to the strong pressure gradient that changes the westerly winds to southerly winds. This temperature gradient reverses in the winter,
when the oceans are warm and the High Himalaya is cold.






\item{What is past literature on rainfall in monsoon season?}

\item{What about clouds and fog? Especially when you are here?}

\item{What are the annual mean temperatures at different elevations?}

\item{What is the lapse rate like?}

\item{What is the vegetation like?}

\item{What is the land use like?}

\item{Make sure to insert a description of the catchment: catchment, Area, mean slope, mean elevation, elevation range, land cover type, geology, Lat Long range}

\begin{table}[h!]
\centering
\begin{small}
\begin{tabular}{p{2cm} p{1cm} p{1cm} p{1.8cm} p{1.8cm} p{2cm} p{1.5cm} p{2.8cm}}
\hline
\textbf{Catchment} & \textbf{Area} & \textbf{Mean Slope} & \textbf{\phantom{hi}Mean \phantom{hel} Elevation} & \textbf{\phantom{}Elevation \phantom{hi}Range} & \textbf{Land Cover \phantom{iee}Type} & \textbf{Geology} & \textbf{\phantom{hiidii}Location \phantom{oideo}Range} \\ 
                   & (km\(^2\))    & \phantom{h}(\%)                & \phantom{hdii}(m)                     & \phantom{hifi}(m)                      &                         &                   & \phantom{hiiiiiiiiii}(DD)             \\ \hline
                   &     \phantom{h}325          &                     &                         &    786 - 5697                     &                         &                   &      85.441 - 85.601 E                  \\
                   &               &                     &                         &                          &                         &                   &      27.822 - 28.157 N                 \\ \hline
\end{tabular}
\end{small}
\caption{Catchment characteristics of the study area.}
\label{tab:catchment_characteristics}
\end{table}


% elevation range from DEM
% area from DEM, and software to figure out catchment area

\end{itemize}