
\section{Study Area}
% Provide a description of the sampling methods, field area, and geology.
% - Include maps or figures if necessary.

% want table with catchment characteristics
% also remember to include a table with abbreviations

\begin{itemize}

\item{What is the lat long extent of the study area?}

85.441 - 85.601 E, 27.822 - 28.157 N

\item{What is the elevation range of the catchment?}

\item{What is the area that the river drains? How do you calculate that (i.e. DEM)?}

\item{What is the geology of the area?}

\item{What is the climate of the area? what is it influenced by? i.e. monsoon, etc.}

\item{What is past literature on rainfall in monsoon season?}

\item{What about clouds and fog? Especially when you are here?}

\item{What are the annual mean temperatures at different elevations?}

\item{What is the lapse rate like?}

\item{What is the vegetation like?}

\item{What is the land use like?}

\item{Make sure to insert a description of the catchment: catchment, Area, mean slope, mean elevation, elevation range, land cover type, geology, Lat Long range}

\begin{table}[h!]
\centering
\begin{small}
\begin{tabular}{p{2cm} p{1cm} p{1cm} p{1.8cm} p{1.8cm} p{2cm} p{1.5cm} p{2.8cm}}
\hline
\textbf{Catchment} & \textbf{Area} & \textbf{Mean Slope} & \textbf{\phantom{hi}Mean \phantom{hel} Elevation} & \textbf{\phantom{}Elevation \phantom{hi}Range} & \textbf{Land Cover \phantom{iee}Type} & \textbf{Geology} & \textbf{\phantom{hiidii}Location \phantom{oideo}Range} \\ 
                   & (km\(^2\))    & \phantom{h}(\%)                & \phantom{hdii}(m)                     & \phantom{hifi}(m)                      &                         &                   & \phantom{hiiiiiiiiii}(DD)             \\ \hline
                   &     \phantom{h}325          &                     &                         &    786 - 5697                     &                         &                   &      85.441 - 85.601 E                  \\
                   &               &                     &                         &                          &                         &                   &      27.822 - 28.157 N                 \\ \hline
\end{tabular}
\end{small}
\caption{Catchment characteristics of the study area.}
\label{tab:catchment_characteristics}
\end{table}


% elevation range from DEM
% area from DEM, and software to figure out catchment area

\end{itemize}