

\section{Discussion}
% Explain how to interpret the data.
% - Divide the analysis into different geological provinces.
% - Discuss the characteristics of each province.




Andermann et al, 2012

They also talk about a modelLook at response time, being inveresely proportional to hydraulic diffusivity.
THey assume length scales of 0.5-5km, and typical values of time to be about 45 days.

They are looking at very large discharges, order 5000 m3/s




\bsk

Also want the Maher and Chamberlain rate constant explanation compared to normal data

\bsk

Want to input the beginning steps of the derivation from Maher and Chamberlain, 2014's model. Eg the solution to a heterogeneous irreversible advection reaciton equation



\bsk
\[
\ttmath{C = \frac{C_0}{1 + D} + C_{eq} \cdot \frac{D}{1 + D}}
\]
\[
\ttmath{D = \frac{\tau \cdot D_w}{q}}
\]
\[
\ttmath{D_w = \frac{L \cdot \phi}{T_{eq}}}
\]
\[
\ttmath{L = q \cdot t}
\]
\[
\ttmath{T_{eq} = \frac{C_{eq}}{R \cdot f}}
\]
\[
\ttmath{R = \rho \cdot k \cdot A \cdot X}
\]
\[
\ttmath{k = 8.7 \cdot 10^{-6}  \; mol/m^2/yr}
\]
\[
\ttmath{k = 8.7 \; \mu mol/m^2/yr}
\]
\subsection{Step 1: Rearrange for $D$}
Multiply through by $(1 + D)$:
\[
C \cdot (1 + D) = C_0 + C_{eq} \cdot D
\]

Distribute $C$:
\[
C + C \cdot D = C_0 + C_{eq} \cdot D
\]

Group terms involving $D$:
\[
C \cdot D - C_{eq} \cdot D = C_0 - C
\]

Factor $D$:
\[
D \cdot (C - C_{eq}) = C_0 - C
\]

Solve for $D$:
\[
D = \frac{C_0 - C}{C - C_{eq}}
\]

\subsection{Step 2: Solve for $t$ Using $D$}
Now substitute $D$ into:
\[
D = \frac{\tau \cdot t \cdot \phi \cdot R \cdot f}{C_{eq}}
\]

Rearranging for $t$:
\[
t = \frac{D \cdot C_{eq}}{\tau \cdot \phi \cdot R \cdot f}
\]

Substitute $D$:
\[
t = \frac{\left(\frac{C_0 - C}{C - C_{eq}}\right) \cdot C_{eq}}{\tau \cdot \phi \cdot R \cdot f}
\]

Simplify:
\[
t = \frac{(C_0 - C) \cdot C_{eq}}{(C - C_{eq}) \cdot \tau \cdot \phi \cdot R \cdot f}
\]

\subsection{Conclusion}
The expression for $t$ is:
\[
t = \frac{(C_0 - C) \cdot C_{eq}}{(C - C_{eq}) \cdot \tau \cdot \phi \cdot k \cdot M_{in} \cdot f}
\]
Where:
\begin{itemize}
    \item $C_0$: Initial concentration.
    \item $C$: Current concentration.
    \item $C_{eq}$: Equilibrium concentration.
    \item $\tau$: Characteristic timescale.
    \item $\phi$: Porosity.
    \item $R$: Reaction rate term.
    \item $f$: Scaling factor.
\end{itemize}
