
\section{Introduction}
%%SHARP SCIENTIFIC LANGUAGE


% WHAT IS THE PROBLEM?
\subsection{Project Outline}

Silicate weathering, whereby silicate minerals are dissolved by carbonic acid, 
sequesters atmospheric CO$_2$ over long (10$^5$ year) timescales, influencing global climate regulation.
The Himalayan mountain range spans more than 590,000 km$^2$, and is the source of major rivers, including the Ganges, Brahmaputra, and Indus. 


\bsk

In highly erosive regions where the supply of silicate minerals far exceeds the weathering rate, silicate weathering reactions are thought to be sensitive to climate (Stallard \& Edmund 1983 JGR, West et al, 2005) The dissolution kinetics of the largely silicate rocks in these catchments are sensitive to temperature and runoff because the weathering reactions have not gone to completion. Silicate weathering in the Himalayas as a result of their uplift and erosion in the Cenozoic may have contributed significantly to the global 
cooling over the past 40 million years %[reword] 
(Raymo and Ruddiman, 1992; West et al, 2005, Kump et al, 2000 Ann Rev Earth Sci). 
Thus, it is this sensitivity to climate as well as their large size that makes them important to study, from both a scientific and practical perspective. More recent models have proposed silicate weathering is more sensitive to the hydrological cycle, than to temperature [cite]. There are a range of flow paths with different residence times, and rock is exhumed through these flow paths with a much longer time constant than the water in the flow paths.

\bsk

However, there remain a number of unknowns in the weathering fingerprints of these catchments, namely the residence time of the water,
flow path direction and length, rate of reaction, and extent to which equilibrium is reached.
Understanding residence time in particular is important because the geochemical reactions that are used to quantify weathering 
(and the biogeochemical ones too) are time-dependent.
Residence time will also reflect the variety of flow routes within a catchment, and help to constrain hydrological models. In this contribution the flow paths and residence times of water will be investigated using the chemical weathering products of spring waters from a highly monitored Himalayan catchment. This will not only provide better constraints on reaction rates of silicate mineral dissolution reactions in the field, but also a greater understanding of the role of hydrology in providing a climate-sensitive negative feedback between atmospheric CO$_2$ and silicate mineral dissolution
An added benefit will also be provided via a greater understanding of Himalayan water supplies which are essential for billions of people (Ives and Messerli, 1989). 

%can always pinch from this and write to abstract
    
\newpage

\begin{tcolorbox}[
    colback=customcolor, % Use the defined custom color for background
    colframe=white,      % Set the frame color to white (invisible)
    sharp corners,       % Straight edges
    boxrule=0pt,         % No border width
    breakable,           % Allow the box to break into multiple pages (if needed)
    width=\dimexpr\textwidth+2cm\relax, % Make the box wider than \textwidth
    enlarge left by=-1cm,   % Shift box left to center the extra width
    leftrule=0mm,        % No left border
    rightrule=0mm,       % No right border
    toprule=0mm,         % No top border
    bottomrule=0mm       % No bottom border
]
\textbf{\Large Box 1}
\vspace{-3mm}
\myline\\
\textbf{\Large Chemical Weathering}
\vspace{-3mm}
\myline\\
{\footnotesize
As water passes through the subsurface, it interacts with rock. This causes the addition of solute species to water, and the formation of stable secondary minerals through the dissolution of primary minerals formed at different pressure and temperature. Dissolved CO$_2$ derived from the atmosphere present in rainfall makes it slightly acidic. This acidity is further increased by the presence of decomposition of organic matter and CO$\ttmath{_2}$ production from organic activity in the soil.
\[
\ttmath{H_2O + CO_2 \rightarrow H_2CO_3 \rightarrow HCO_3^- + H^+}
\]

Carbonic Acid Weathering of Carbonate - Net Zero
    
    \begin{center}
    
    Short Term [timescale]:
    \[
    \ttmath{CaCO_3 + CO_2 + H_2O = Ca^{2+} + 2HCO_3^-}
    \]
    
    Long Term [timescale]:
    \[
    \ttmath{Ca^{2+} + 2HCO_3^- = CaCO_3 + CO_2 + H_2O}
    \]

    \end{center}
    
    
Carbonic Acid Weathering of Silicate - Net CO$\ttmath{_2}$ drawdown

    \begin{center}

    Short Term [timescale]:
    \[
    \ttmath{2CO_2 + 3H_2O + CaAl_2Si_2O_8 = Ca^{2+} + 2HCO_3^- + Al_2Si_2O_5(OH)_4}
    \]
    
    Long Term [timescale]:
    \[
    \ttmath{Ca^{2+} + 2HCO_3^- = CaCO_3 + CO_2 + H_2O}
    \]
    
    \end{center}

The rate of weathering is dependent on the mineralogy of the rock. Different minerals weather at different rates (from Shand et al, 1999: quartz > albite > mafic silicates > anorthite > carbonates), so the most reactive minerals will contribute disprportionately to the solute load of the water. [Something about Tipper 2006 and carbonates during monsoon?? somewhere]. Note that here, only the weathering of carbonic acid is considered. Sulfuric acid is also often considered a big player in weathering, but its impact is not considered in this study because the pyrite deposits required for its formation are not present in lithological studies of the Melamchi region [follow up].
\[
\ttmath{4FeS_2 + 15O_2 + 14H_2O \rightarrow 4Fe(OH)_3 + 8H_2SO_4}
\] 

Mineral reactions are also time-dependent, so the longer that water spends in contact with the rock, the higher the degree of completion of a chemical reaction [can mention incongruent dissolution whereby only part of the mineral dissolves?]. Current models of silicate and carbonate weathering (the two dominant lithologies considered) do not generally consider underground flow paths in their carbon flux estimates (Gaillardet et al, 1999 and others). Hence, a potentially underestimated part of the carbon cycle is this underground weathering.

}
\end{tcolorbox}


\newpage

%\section{Literature Review}



% DISSOLUTION RATES:

% \item{Kump et al, 2000: The relative rates of dissolution of silicate minerals are similar in field and laboratory but they differ by orders of magnitude. Chemical erosion rates are dependent on how intense the hydrological cycle is}

% \item{White and Brantley, 2003: Field conditions are different to lab conditions. Processes "extrinsic" to the weathering conditions cannot be replicated in the lab}

% \item{Kampman et al, 2009: Correlation between delta G and rate of reaction when comparing lab and field rates - they plot in different places, the field ones being closer to equilibrium.}

% \item{West et al, 2005: } - BRIDGE

% CONTROLS ON WEATHERING - LINK FROM West:

% \item{Stallard and Edmond, 1983: Geological variations lead to differences in soil composition, landscape features, vegetation, and climate. Topography also influences soil properties. The contribution of weathering from different lithologies should be proportional to their area within the catchment.}

% \item{Pedrazas et al, 2021: Describe two weathering fronts, shallow with extensive fracturing and deep with open fractures. Weathering fraction scales with hillslope length}

% \item{Singh et al, 1987: Increase in porosity for the same decrease in density for sanstone > porphyritic gneiss > quartzite > pegmatite > amphibolite > biotite gneiss > basalt > dolerite}

% \item{Marques et al, 2009: Porosity increases as the rock becomes more weathered, as does the water content. Maybe you can put their weathering index there}

% \item{David et al, 1994: Low porosity rock have porosity less than 5\%. High porosity rocks have porosity greater than 15\%. Relationship between permeability and porosity and depth is exponential, link to Mike}

% \item{Gaillardet et al, 1999: 60 largest rivers in the worlkd. Aims to fit an inverse model looking at silicate weathering for Co2 consumption. bickle on why inverse models do not work.... Carbonate weathering has no effect over million year timescales on atmospheric CO2. low Ca/Na molar ratios are expected in dissolved load of rivers draining silicates }

% \item{Medwedeff et al, 2021: Weathering controls shallow bedrock strength more so than mineral or textural differences between the metamorphic lithologies. Weathering also varies systematically with vegetation and topography. Weathering decreases above the tree line (though we do not have data for this).}


% PRECIPITATION EFFECTS:

% \item{Andermann et al. (2012) show that anticlockwise hysteresis loops of precipitation against discharge (include basic schematic) suggest that there is a 3 month lag in the response of the river to precipitation.  The delay in river discharge is a topic of debate. Andermann et al. (2012) propose that at lower elevations, it is more likely due to groundwater storage of precipitation in the fractured basement}

% \item{Bookhagen and Burbank (2010) suggest that the delay of precipitation and discharge may be due to the response of glaciers at higher elevations. They also suggest evaportranspiration has a low impact on the hydrological budget of the Himalayas, less than 10\%.}

% \item{McGuire et al, 2005: Catchments with little glacial input show the same delay. Residence time of groundwater can hence be used to quantify this delay and nature of its origin}


\subsection{Lithological controls on weathering}

Differences in lithology are thought to affect weathering. Geological differences lead to differences in soil composition, landscape features, vegetation, and climate which in turn affect the rates of reaction. Logically, the contribution of one lithology to weathering is correlated to its spatial extent in the catchment (Stallard and Edmond, 1983). Porosities vary widely across a catchment depending on the rock type encountered (Singh et al, 1987; David et al, 1994). Porosity also increases as a rock becomes more weathered (Marques et al, 2009). Weathering regimes can be classified as either transport-limited or kinetically limited (West et al., 2005). West et al. distinguish the two regimes by the rate of erosion in the catchment. In low erosion rate settings, weathering is transport-limited due to limited mineral supply. Weathering here is therefore proportional to the material eroded. In high erosion rate settings, weathering is kinetically limited due to an abundant mineral supply. Rapidly eroding catchments like Melamchi are therefore likely kinetically limited. Soil properties and topography are used to identify different "weathering regimes" in the subsurface (Pedrazas et al, 2021). Indeed, bedrock strength is thought to be more dependent on weathering than mineral or textural differences between the metamorphic lithologies in the Himalayas (Medwedeff et al, 2021).

\subsection{Dissolution rate findings}

There is general consensus in the scientific community that rates of dissolution of minerals are different depending on whether they are measured in the field or in a laboratory. White and Brantley (2003) explain this difference by denoting 'extrinsic' qualities that are variable in the field, such as permeability and mineral/fluid ratios. The rate of dissolution of a mineral has also been linked to the free energy of the system, with laboratory rates being calculated significantly more far away from equilibrium (a higher delta G) than field rates (Kampman et al, 2009). This implies that field localities are closer to equilibrium than laboratory-derived rates might suggest. Add Maher et al, 2009 and why they think lab and field are different.

\subsection{Response to monsoonal precipitation}
What makes the Himalayas unique is also what makes them difficult to model, namely the monsoon. This is characterised by a strong seasonal reversal of winds, which brings heavy rainfall to the region during the summer months, and dry conditions during the winter (Bookhagen and Burbank, 2010). The monsoon brings a large amount of precipitation to the region. Andermann et al. (2012) show that anticlockwise hysteresis loops of precipitation against discharge (include basic schematic) suggest that there is a 3 month lag in the response of the river to precipitation.  The delay in river discharge is a topic of debate. Andermann et al. (2012) propose that at lower elevations, it is more likely due to groundwater storage of precipitation in the fractured basement. Bookhagen and Burbank (2010) suggest that the delay of precipitation and discharge may be due to the response of glaciers at higher elevations. They also suggest evaportranspiration has a low impact on the hydrological budget of the Himalayas, less than 10\%. Other studies have found that catchments with little glacial input show the same delay, suggesting that the former hypothesis may be more relevant to this discussion (McGuire et al, 2005). The residence time of groundwater can hence be used to quantify this delay and nature of its origin.



% PRECIPITATION AND CLIMATE:

% \item{Acharya et al, 2020: Higher elevations during the monsoon receive lots of water from the southerly winds rather than from the westerlies. Makes sense. Also suggests most of the water input comes from the top quarter or so of the catchment.}

% \item{Wen et al, 2012: Southern Himalaya have a large altitude contrast over a short distance. Isotopic lapse rate is smaller in the monsoon season and higher in the non monsoon seasonl, implying a role of the monsoon precipitation in the lower lapse rate}

% \item{Kattel et al, 2012: They can model the lapse rate in the Himalayas. The large range in altitude clearly leads to a large spatial variation in temperature. Climate in Nepal is tropical to alpine in s very short x-y distance. They state that the foothill slopes of the Himalayas receive the most rain, which disagrees with Acharya et al, 2020.}

% \item{Panthi et al, 2015: The arrival date for the monsoon in Nepal has not changed bu the withdrawal is getting later. Precipitation is more intense on a per day basis. This is bad for the crops in winter season because of lack of moisture and obviously causes more disasters and floods} LINK TO DISASTER

% \item{Baniya et al, 2012: Flooding in Melamchi in 2021: precipitation is the main climatic driver of flooding, and increasing temperature leads to higher magnitude floods} LINK TO DISASTER



% DISASTER EFFECTS:

% \item{Graf et al, 2023: the 2015 Gorkha EQ changed the geomorphology of the Melamchi catchment. The landslides that followed contributed significantly to the 13 milliom m3 sediment deposited during the later 2021 Melamchi flood. The EQ was important but only tipped an already unstable system.}

% The monsoon brings a large amount of water in the form of precipitation to the region, which reacts with the CO$_2$ in the air to make carbonic acid. This subsequently reacts and dissociates with the silicate minerals, drawing down atmospheric CO$_2$ (see some section with the full equations). [Something about Illien and how it is linked to groundwater flow]


\subsection{Study aims}

% WHAT DOES THIS STUDY DO?

In the present study, spring and rain samples from the Melamchi catchment are used as a case 
study to investigate the weathering rates in a rapidly eroding catchment. The sample dataset consists of 372 samples spanning four field campaigns over three years (2021-2024), as well as more recent year-long bi-weekly timeseries data from stream and spring samples in sites across the catchment. [See map] 
As Tipper et al, 2006 writes, studying small catchments gives the opportunity to attribute large changes in water chemistry to seasonal climate changes like the monsoon. (More in Area)


% \item WHAT DOES THIS STUDY SHOW IN THEIR RESULTS? % \item WHAT HAVE PREVIOUS MODEL STUDIES SHOWN?
 
\bsk

This study considers major and trace ion concentrations, alkalinity, and radiogenic strontium isotopes from the Melamchi catchment. This study shows that there are systematic variations in the chemical composition of the water along and away from the ridge. These can be explained through lithological differences and chemical weathering respectiuvely. Estimation of the carbon flux in the groundwater yields ******?. Residence time calculations determined using rate constants close to equilibrium give ages of 10-25 years. This is in agreement with previous studies which calculated residence times using gas ages (Atwood et al [expand on this]). Previous studies have linked residence time to topography, such that areas with a small topographic gradient evolve to a larger residence time and vice versa. (McGuire et al, 2005). A similar relationship is found in this study.

 
\bsk
% \item WHAT ARE WE DOING DIFFERENTLY? WHAT ARE THE ADVANTAGES OF DOING THIS DIFFERNTLY?

So far, papers modelling the evolution of weathering have not considered catchment-based data due to the non-ideal setting. This study aims to bridge that gap by applying simplified models to a highly monitored catchment. The hope is that this approach will help to give a first-order estimation on parameters that are not easily obtained with the chemistry of a system alone. This report aims to join together studies looking at the same problem from different disciplines.





% \item WHAT ARE THE QUESTIONS WE HOPE TO ANSWER?






% Summary of the key questions (commented):
% 1. What main challenge does this study address?
% 2. What prior work and prevailing perspectives exist?
% 3. What does this investigation aim to accomplish?
% 4. Which results are highlighted by this study?
% 5. How have previous models approached the problem?
% 6. In what ways does this work differ from existing studies?
% 7. Why is this approach potentially more beneficial?
% 8. Which specific questions does this study seek to answer?
