
\section{Introduction}
%%SHARP SCIENTIFIC LANGUAGE

\subsection{Aims and Motivation}

Silicate weathering, whereby silicate minerals are dissolved by carbonic acid, 
sequesters atmospheric CO$_2$ over long (10$^6$ year) timescales, influencing global climate regulation. The Himalayan mountain range spans more than 590,000 km$^2$, and is the source of major rivers, including the Ganges, Brahmaputra, and Indus. It is therefore a key region for understanding the global carbon cycle.

\bsk

In highly erosive regions where the supply of silicate minerals far exceeds the weathering rate, silicate weathering reactions are thought to be sensitive to climate (Stallard \& Edmund 1983 JGR, West et al, 2005) The dissolution kinetics of the largely silicate rocks in these catchments are thought to be sensitive to temperature and runoff because the weathering reactions have not gone to completion. Silicate weathering in the Himalayas as a result of their uplift and erosion in the Cenozoic may have contributed significantly to the global 
cooling over the past 40 million years (Raymo and Ruddiman, 1992; West et al, 2005, Kump et al, 2000 Ann Rev Earth Sci). Thus, it is this reported sensitivity to climate as well as their large size that makes them important to study, from both a scientific and practical perspective. However, more recent models have proposed silicate weathering is more sensitive to the hydrological cycle, than to temperature (Maher, 2011). The motivation of this project is to model these two hypotheses using groundwater samples obtained from a Himalayan catchment to determine the greatest control on weathering. This comparison will also prove useful for discussions of the nature of rates of reaction in real catchments. An added benefit will also be provided via a greater understanding of Himalayan water supplies which are essential for billions of people (Ives and Messerli, 1989). 


\subsection{Study Dataset and Modeling}

In the present study, spring and rain samples from the Melamchi catchment are used as a case 
study to investigate the weathering rates in a rapidly eroding catchment. The sample dataset consists of 372 samples spanning four field campaigns over three years (2021-2024), as well as more recent year-long bi-weekly timeseries data from stream and spring samples in sites across the catchment. This dataset comprises major and trace ion concentrations, alkalinity, and radiogenic strontium isotopes from the Melamchi catchment. Spatial analysis shows that there are systematic variations in the chemical composition of the water along and away from the ridge. The possible controls on these variations, namely temperature, flow path length, lithology, dissolution rates, and evapotranspiration cannot be distinguished. As a result, the one-dimensional reactive transport models will hone in on one profile perpendicular to the ridge (Traverse 3). 

\bsk

Studying small catchments gives the opportunity to attribute large changes in water chemistry to seasonal climate changes like the monsoon. (Tipper et al, 2006). Two models are evaluated and pitted against each other to determine the greatest control on weathering in the Himalayas. The first model agrees with the initial 'null' hypothesis, that higher temperatures imply greater weathering intensities, and so more primary mineral dissolution (Fontorbe et al, 2013). The second model suggests that the primary control on weathering products is discharge, under the condition that fluid residence times are long enough so that fluids reach equilibrium with the solids (Maher, 2011).


\subsection{Weathering Fingerprints}

The weathering fingerprints of these catchments contain many unknowns, namely the residence time of the water, flow path direction and length, rate of reaction, and extent to which equilibrium is reached. Understanding residence time in particular is important because the geochemical reactions that are used to quantify weathering (and the biogeochemical ones too) are time-dependent; longer residence times promote greater solute accumulation in the water (Berner, 1978). These geochemical reactions are also controlled by the reaction rate which is thought to vary as equilibrium is approached (Maher, 2011). Therefore, an understanding of residence time will provide insight into the rate of reaction, and vice versa. Residence time will also reflect the variety of flow routes within a catchment, and help to constrain hydrological models.

\bsk

From the model comparison will come a better understanding of fluid residence times in Himalayan catchments, for which tracer data is already commonly used to infer how long a water packet spends in the subsurface (Atwood et al, 2021). Previous studies on Melamchi have used CFC and SF$_6$ gases to determine a mean age on the order of ten years for groundwater at the base of the catchment ridge (Atwood et al, 2021). Using the chemical composition of the water will provide a different way of obtaining residence times, and give a benchmark for the tracer data, which is often purported to be limited in its application (McCallum et al, 2015). If the residence time of a particular water packet is long enough, it will reach chemical equilibrium, meaning the free energy of the system will be close to zero. Comparison of these residence times with estimates of free energy will provide a test on the validity of the two models and their assumptions.


\newpage

\section{Literature Review}

\subsection{Defining Weathering}

As water passes through the subsurface, it interacts with rock. This causes the addition of solute species to water, and the formation of stable secondary minerals through the dissolution of primary minerals formed at different pressure and temperature. Dissolved CO$_2$ derived from the atmosphere present in rainfall makes it slightly acidic:\\
\begin{equation}
H_2O + CO_2 \rightarrow H_2CO_3 \rightarrow HCO_3^- + H^+
\end{equation}\\
Once the rainfall enters the soil as groundwater, the acidicity is further increased by the presence of decomposition of organic matter, and CO$_2$ production from organic activity in the soil.

\bsk

The rate of weathering is dependent on the mineralogy of the rock. Different minerals weather at different rates: quartz > albite > mafic silicates > anorthite > carbonates, so the most reactive minerals will contribute disprportionately to the solute load of the water (Shand et al, 1999). In the Melamchi catchment, only the weathering of carbonic acid is considered. Sulfuric acid is also often considered a big player in weathering, but its impact is not considered in this study because the pyrite deposits required for its formation are not present in lithological studies of the Melamchi region.

\newpage

\begin{tcolorbox}[
    colback=customcolor, % Use the defined custom color for background
    colframe=white,      % Set the frame color to white (invisible)
    sharp corners,       % Straight edges
    boxrule=0pt,         % No border width
    breakable,           % Allow the box to break into multiple pages (if needed)
    width=\dimexpr\textwidth+2cm\relax, % Make the box wider than \textwidth
    enlarge left by=-1cm,   % Shift box left to center the extra width
    leftrule=0mm,        % No left border
    rightrule=0mm,       % No right border
    toprule=0mm,         % No top border
    bottomrule=0mm       % No bottom border
]
\textbf{\Large Box 1: Chemical Weathering Reactions}
\vspace{-3mm}
\myline\\
{\footnotesize
Chemical weathering reactions are dependent on the lithology and acid content of the water. Below are the primary reactions that characterise carbonic acid weathering worldwide.

\bsk

\textbf{Carbonic Acid Weathering of Carbonate}\\
This reaction has no effect on atmospheric CO$_2$ levels in the long term.

    \begin{center}
    
    Short Term (10$^3$ years):
    \begin{equation}
    CaCO_3 + CO_2 + H_2O \rightarrow Ca^{2+} + 2HCO_3^-
    \end{equation}
    
    Long Term (10$^6$ years):
    \begin{equation}
    Ca^{2+} + 2HCO_3^- \rightarrow CaCO_3 + CO_2 + H_2O
    \end{equation}

    \end{center}
    
    
\textbf{Carbonic Acid Weathering of Silicate}\\ This produces net CO$\ttmath{_2}$ drawdown. Reactions are written for a generic Ca-rich plagioclase mineral.

    \begin{center}

    Short Term (10$^3$ years):
    \begin{equation}
    2CO_2 + 3H_2O + CaAl_2Si_2O_8 \rightarrow Ca^{2+} + 2HCO_3^- + Al_2Si_2O_5(OH)_4
    \end{equation}
    
    Long Term (10$^7$ years):
    \begin{equation}
    \ttmath{Ca^{2+} + 2HCO_3^- \rightarrow CaCO_3 + CO_2 + H_2O}
    \end{equation}
    
    \end{center}

}
\end{tcolorbox}


\subsection{Geological Controls on Weathering}

Differences in lithology are thought to affect weathering. Geological differences lead to differences in soil composition, landscape features, vegetation, and climate which in turn affect the rates of reaction. Logically, the contribution of one lithology to weathering is correlated to its spatial extent in the catchment (Stallard and Edmond, 1983). Porosities vary widely across a catchment depending on the rock type encountered (Singh et al, 1987; David et al, 1994). Porosity also increases as a rock becomes more weathered (Marques et al, 2009). Weathering regimes can be classified as either transport-limited or kinetically limited (West et al., 2005). West et al. distinguish the two regimes by the rate of erosion in the catchment. In low erosion rate settings, weathering is transport-limited due to limited mineral supply. Weathering here is therefore proportional to the material eroded. In high erosion rate settings, weathering is kinetically limited due to an abundant mineral supply. Rapidly eroding catchments like Melamchi are therefore likely kinetically limited. Soil properties and topography are used to identify different "weathering regimes" in the subsurface (Pedrazas et al, 2021). Indeed, bedrock strength is thought to be more dependent on weathering than mineral or textural differences between the metamorphic lithologies in the Himalayas (Medwedeff et al, 2021).



\subsection{Reaction Rates in Natural and Laboratory Settings}

Rates of reaction during weathering comprise both dissolution and precipitation, and chemical equilibrium is defined as that state where these are balanced and equal. Rates of reaction are thought to be different depending on whether they are measured in the field or in a laboratory (Maher et al., 2009). This difference has been explained by denoting 'extrinsic' qualities that are variable in the field, such as permeability and mineral/fluid ratios (White and Brantley, 2003). The rate of reaction of a system has also been linked to the free energy of the system, with laboratory rates being calculated significantly further away from equilibrium than field rates (Kampman et al, 2009). This implies that field localities are closer to equilibrium than laboratory-derived rates might suggest. 



\subsection{Response to Monsoonal Precipitation}

The Indian Summer Monsoon (ISM) is characterised by a strong seasonal reversal of winds, which brings heavy rainfall to the region during the summer months, and dry conditions during the winter (Bookhagen and Burbank, 2010). The monsoon brings a large amount of precipitation to the region. Oxygen isotopes suggest most of the precipitation occurs in the higher elevation parts of the catchment, and this is supported by remotely sensed rainfall estimates in the region (Acharya et al, 2020; Bookhagen and Burbank, 2010). Seasonal variation in rainfall is thought to relate to different hydrological regimes, whereby river discharge and precipitation are 'coupled' when there is a significant enough amount of water to recharge the groundwater system. (Illien et al, 2021) The seasonal variation in precipitation therefore also translates to a variation in runoff, whereby this is twelve times stronger during the monsoon than during the dry season (Sharma, 1997). 

\bsk

Andermann et al. (2012) report anticlockwise hysteresis loops of precipitation against discharge (include figure?), and suggest that there is a three month lag in the response of the river to precipitation.  The delay in river discharge is a topic of debate. Andermann et al. (2012) propose that at lower elevations, it is more likely due to groundwater storage of precipitation in the fractured basement. Bookhagen and Burbank (2010) suggest that the delay of precipitation and discharge may be due to the response of glaciers at higher elevations. They also suggest evaportranspiration has a low impact on the hydrological budget of the Himalayas, less than 10\%. Other studies have found that catchments with little glacial input show the same delay, suggesting that the former hypothesis may be more relevant to this discussion (McGuire et al, 2005). The residence time of groundwater can hence be used to quantify this delay and nature of its origin, given that rain is the main source of recharge to the groundwater system (Illien et al, 2021). 


\subsection{Effects of a Changing Climate}

Changes in climate contribute to changes in the monsoonal system dynamics. The start of the monsoon has not changed in Nepal, but the end has been delayed. This has led to more intense precipitation on a per day basis, which is detrimental for crops in the winter season due to lack of moisture. Intense precipitation is also considered the main climatic cause of flooding (Panthi et al, 2015; Baniya et al, 2012). 

\bsk

"One-off" landslide events transport as much as four times the flux of sediment deposited in the valley in a year (Chen C et al., 2023). These events are thought to be increasing in frequency over recent years as a result of climate change, increasing the erosion rate in these areas (Adhikari et al, 2023). In particular, effects of a flash flood in 2021 are still visible in the area, with damage done to several bridges and hundreds of families. 



% Summary of the key questions (commented):
% 1. What main challenge does this study address?
% 2. What prior work and prevailing perspectives exist?
% 3. What does this investigation aim to accomplish?
% 4. Which results are highlighted by this study?
% 5. How have previous models approached the problem?
% 6. In what ways does this work differ from existing studies?
% 7. Why is this approach potentially more beneficial?
% 8. Which specific questions does this study seek to answer?
