
\section{Introduction}
%%SHARP SCIENTIFIC LANGUAGE


% WHAT IS THE PROBLEM?

Silicate weathering, whereby silicate minerals are dissolved by carbonic acid 
sequesters atmospheric CO$_2$ over long (10$^7$ year, check) timescales, influencing global climate regulation.
The Himalayan mountain range spans more than 590,000 km$^2$, and is the source of major rivers, including the Ganges, Brahmaputra, and Indus. In highly erosive regions where the supply of silicate minerals far exceeds the weathering rate, silicate weathering reactions are thought to be sensitive to climate (Knight et al, 2024) The dissolution kinetics of the largely silicate rocks in these catchments are sensitive to temperature and runoff. Silicate weathering in the Himalayas as a result of their uplift and erosion in the Cenozoic is therefore thought to have contributed significantly to the global 
cooling over the past 40 million years %[reword] 
(Raymo and Ruddiman, 1992; West et al, 2005). 
Thus, it is this sensitivity to climate as well as their large size that makes them important to study, from both a scientific and practical perspective. More recent models have proposed silicate weathering is more sensitive to the hydrological cycle, than to temperature [cite].
However, there remain a number of unknowns in the weathering fingerprints of these catchments, namely the residence time of the water,
flow path direction and length, rate of reaction, and extent to which equilibrium is reached.
Understanding residence time in particular is important because the geochemical reactions that are used to quantify weathering 
(and the biogeochemical ones too) are time-dependent.
Residence time can also reveal the variety of flow routes within a catchment, and help to constrain hydrological models. In this contribution the flow paths and residence times of water will be solved using the chemical weathering products of spring waters from a highly monitored Himalayan catchment. This will not only provide a better understanding of the field based reaction rates of silicate mineral dissolution reactions, but also a greater understanding of the role of hydrology in providing a climate-sensitive negative feedback between atmospheric CO$_2$ and silicate mineral dissolution
An added benefit will also be provided via a greater understanding of Himalayan water supplies which are essential for billions of people (Ives and Messerli, 1989). 

%can always pinch from this and write to abstract
    
\bsk

% WHAT HAVE PEOPLE DONE SO FAR AND WHAT DO PEOPLE THINK?

What makes the Himalayas unique is also what makes them difficult to model, namely the monsoon system that characterises the region.
The monsoon system is characterised by a strong seasonal reversal of winds, which brings heavy rainfall to the region during the summer months, and dry conditions during the winter.
The monsoon system brings a large amount of water in the form of precipitation to the region, which reacts with the CO$_2$ in the air to make carbonic acid. This subsequently reacts and dissociates with the silicate minerals, drawing down atmospheric CO$_2$ (see some section with the full equations).

\bsk

Work done by Andermann et al. (2012) on the Melamchi catchment in Nepal shows that the discharge of the river is highly seasonal. Anticlockwise hysteresis loops of precipitation against discharge (include basic schematic) suggest that there is a delay in the response of the river to precipitation.  The delay in river discharge is a topic of debate. 
Bookhagen and Burbank (2010) suggest it may be due to glaciers at higher elevations, 
while Andermann et al. (2012) propose that at lower elevations, it is more likely due 
to groundwater storage of precipitation in the fractured basement. Residence time of groundwater can hence be used to quantify this delay and nature of its origin (McGuire et al, 2005).

\bsk

As water passes through the subsurface, it interacts with the rock. This is the basis behind chemical weathering. Mineral reactions are time-dependent, so the longer that water spends in contact with the rock, the higher the degree of completion of a chemical reaction\footnote{Obviously there is secondary precipitation occuring as well as mineral dissolution}. Current models of silicate and carbonate weathering (the two dominant lithologies considered) do not generally consider underground flow paths in their carbon flux estimates (Gaillardet et al, 1999 and others) [Something about Tipper 2006 and carbonates during monsoon?? somewhere]. Hence, a potentially underestimated part of the carbon cycle is this underground weathering.

\bsk

Weathering regimes can be classified as either transport-limited or kinetically limited. West et al. (2005) distinguishes the two regimes by the rate of erosion in the catchment. In low erosion rate settings, weathering is transport-limited due to limited mineral supply. Weathering here is therefore proportional to the material eroded. In high erosion rate settings, weathering is kinetically limited due to an abundant mineral supply. Rapidly eroding catchments like Melamchi are therefore likely kinetically limited. Field evidence supports this claim, with landslides being frequent during the monsoon (Baniya et al., 2010).

    \bsk


% WHAT DOES THIS STUDY DO?

In the present study, spring and rain samples from the Melamchi catchment are used as a case 
study to investigate the weathering rates in a rapidly eroding catchment. The sample dataset consists of 372 samples spanning four field campaigns over three years (2021-2024), as well as more recent year-long bi-weekly timeseries data from stream and spring samples in sites across the catchment. [See map] 
As Tipper et al, 2006 writes, studying small catchments gives the opportunity to attribute large changes in water chemistry to seasonal climate changes like the monsoon. (More in Area)


% \item WHAT DOES THIS STUDY SHOW IN THEIR RESULTS? % \item WHAT HAVE PREVIOUS MODEL STUDIES SHOWN?
 
\bsk

This study considers major and trace ion concentrations, alkalinity, and radiogenic strontium isotopes from the Melamchi catchment. This study shows that there are systematic variations in the chemical composition of the water along and away from the ridge. These can be explained through lithological differences and chemical weathering respectiuvely. Estimation of the carbon flux in the groundwater yields ******?. Residence time calculations determined using rate constants close to equilibrium give ages of 10-25 years. This is in agreement with previous studies which calculated residence times using gas ages (Atwood et al [expand on this]). Previous studies have linked residence time to topography, such that areas with a small topographic gradient evolve to a larger residence time and vice versa. (McGuire et al, 2005). A similar relationship is found in this study.


% \item WHAT ARE WE DOING DIFFERENTLY?




% \item WHAT ARE THE ADVANTAGES OF DOING THIS DIFFERNTLY?




% \item WHAT ARE THE QUESTIONS WE HOPE TO ANSWER?






% Summary of the key questions (commented):
% 1. What main challenge does this study address?
% 2. What prior work and prevailing perspectives exist?
% 3. What does this investigation aim to accomplish?
% 4. Which results are highlighted by this study?
% 5. How have previous models approached the problem?
% 6. In what ways does this work differ from existing studies?
% 7. Why is this approach potentially more beneficial?
% 8. Which specific questions does this study seek to answer?
