
\section{Introduction}

Silicate weathering, whereby silicate minerals are dissolved by carbonic acid, sequesters atmospheric CO$_2$ over long (10$^6$ year) timescales, and is proposed to be the dominant mechanism by which global climate is controlled on multi-million year timescales. As water passes through the subsurface, it interacts with the surrounding rock. This causes the addition of solute species to the groundwater, and the formation of stable secondary minerals through the dissolution of primary minerals. Large mountain ranges in particular are thought to be most sensitive to weathering (Tipper et al., 2006). The Himalayan mountain range spans more than 590,000 km$^2$, and is the source of major rivers, including the Ganges, Brahmaputra, and Indus. Silicate weathering in the Himalayas as a result of their uplift and erosion in the Cenozoic may have contributed significantly to the global cooling over the past 40 million years (Raymo and Ruddiman, 1992; West et al, 2005, Kump et al, 2000). Clearly, understanding the strongest control on weathering in these key regions for the global carbon cycle is of utmost importance to inform climate policy.

\bsk

 The dissolution kinetics of the silicate rocks in Himalayan catchments are thought to be sensitive to temperature and runoff because the weathering reactions have not gone to completion.  Weathering regimes can be classified as either transport-limited or kinetically limited (West et al., 2005). West et al. (2005) distinguish the two regimes by the rate of erosion in the catchment. In low erosion rate settings, weathering is transport-limited due to limited mineral supply. Here, weathering rate is proportional to the material eroded. In high erosion rate settings, there is an abundant mineral supply and so only kinetics stand in the way of the weathering reaction. Rapidly eroding catchments like Melamchi are likely kinetically-limited (Stallard \& Edmund 1983 JGR, West et al, 2005). In highly erosive regions like Melamchi, silicate weathering reactions are thought to be more sensitive to temperature than runoff. Soil properties and topography are also linked to weathering regimes in the subsurface (Pedrazas et al, 2021). Indeed, bedrock strength is thought to be more dependent on weathering than mineral or textural differences between the metamorphic lithologies in the Himalayas (Medwedeff et al, 2021). Therefore, understanding the controls of weathering can also help to predict the stability of bedrock in rapidly eroding regions. 

\bsk

Reactive transport models are widely used in Earth Sciences to simulate the flow of groundwater through the subsurface (Bethke, 2011). These models can be used to simulate chemical weathering reactions through one-dimensional flow paths based on a few key parameters, which make up the "weathering fingerprint" of a catchment. Models built on different assumptions on weathering control will produce different results. Some of these models are built on a strong temperature control on weathering ("Fontorbe models" from now on, after Fontorbe et al., 2013). More recent models have proposed silicate weathering is more sensitive to runoff than to temperature (Maher, 2011). These ("Maher models" from now on, after Maher, 2011) assume that all weathering paths approach equilibrium. This study looks at Melamchi, a Himalayan catchment north-east of Kathmandu in Nepal, to act as a case study for Himalayan weathering. A proxy is needed for the comparison of the two models, in order to determine which one most accurately represents the weathering in Melamchi.

\bsk

Estimation of residence time from the Maher and Fontorbe models using measured spring chemistry in Melamchi will act as the proxy for weathering control. The weathering fingerprints of these catchments contain many unknowns, namely the residence time of the water, mineral surface areas, rate of reaction, and extent to which equilibrium is reached. Spring chemistry is reflective of the weathering processes that occur in the subsurface, and can therefore be used to estimate these fingerprint parameters. Residence time, also called the advective age, is defined by how long a given water packet spends from rain recharge into groundwater to exiting at a spring (McCallum et al, 2015). Understanding residence time in particular is important because the geochemical and biogeochemical reactions that contribute to the solute load during weathering are time-dependent; generally, longer residence times promote greater solute accumulation in the water (Berner, 1978). These geochemical reactions are also controlled by the reaction rate which is thought to vary as equilibrium is approached (White and Brantley, 2003; Maher, 2011). Therefore, an understanding of residence time will provide insight into how long weathering reactions take place in a given catchment, whether they reach equilibrium, and what this means for the carbon cycle as a whole. Residence time will also inform the effects of drought in a catchment. Long residence times are likely to be more resilient to periods without rainfall (Atwood et al., 2021). Calculating residence time using the spring chemistry in Melamchi will allow the assumptions underlying the two models to be tested, and assess their applicability to a real-world catchment.

\bsk

From the model comparison will also come a better understanding of fluid residence times in Himalayan catchments, for which tracer data is already commonly used to infer how long a water packet spends in the subsurface (Atwood et al, 2021). Previous studies on Melamchi have used CFC and SF$_6$ gases to determine a mean age on the order of ten years for groundwater at the base of the catchment ridge (Atwood et al, 2021) (ref map). Using the chemical composition of the water will provide a different way of obtaining residence times, and give a benchmark for the tracer data, which is often reported to be limited in its application (McCallum et al, 2015). If the residence time of a particular water packet is long enough, the reaction will reach chemical equilibrium, meaning the free energy of the system will be close to zero (Kampman et al., 2013). Comparison of these residence times with separate estimates of free energy derived from the measured concentration of springwater in the catchment will test the validity of the two models and their assumptions. Finally, this will help to inform whether weathering in catchments like Melamchi is most strongly controlled by temperature or runoff.

\bsk

Rates of reaction during weathering comprise both dissolution and precipitation, and chemical equilibrium is defined as that state where these are balanced and equal. The rate of weathering is dependent on the mineralogy of the rock. Different minerals weather at different rates. This also depends on the mineral surface area available for reaction, which is one of the most underconstrained aspects of weathering models (White and Brantley, 2003). The most reactive minerals will disproportionately contribute to the solute load of the water (Shand et al, 1999). Therefore, understanding the primary mineral reaction to model is key when quantifying weathering in a catchment. Rates of reaction are thought to be different depending on whether they are measured in the field or in a laboratory (Maher et al., 2009). This difference has been explained by denoting 'extrinsic' qualities that are variable in the field, such as permeability, mineral/fluid ratios and different surface areas available to react (White and Brantley, 2003). The rate of reaction of a system has also been linked to the free energy of the system, with laboratory rates being calculated significantly further away from equilibrium than field rates (Kampman et al, 2009). This implies that field localities are closer to equilibrium than laboratory-derived rates might suggest. 

\bsk

Differences in lithology are also thought to affect weathering. Geological differences lead to differences in soil composition, landscape features, vegetation, and climate which in turn affect the rates of reaction. Logically, the contribution of one lithology to weathering is at least in part correlated to its spatial extent in the catchment (Stallard and Edmond, 1983). In the Melamchi catchment, only weathering through carbonic acid is considered. Weathering through sulfuric acid also significantly contributes to the global weathering budget, but its impact is not considered in this study because the marine deposits required for its formation are not present in the Melamchi region (Bufe et al., 2021) 

\bsk

Estimation of porosity is essential for understanding the extent of weathering in a catchment. Understanding how open a rock is to water flow and reaction can constrain the reactive transport models used to estimate residence time. Porosities vary widely across a catchment depending on the rock type encountered (Singh et al, 1987; David et al, 1994). Porosity also increases as a rock becomes more weathered (Marques et al, 2009). Note that in the following models, the porosity value is taken to be an average over a given depth in the subsurface. In Earth Sciences, models of fluid flow — whether in the subsurface or deep within the Earth — are typically categorized based on whether the flow occurs through a porous medium or within large open channels (Pedrazas et al, 2021; Maher, 2011; Kelemen et al, 1999; Jackson et al., 2018). This remains an open debate beyond the scope of this study (though note that in later sections flow paths are depicted as "channels" to facilitate the explanation of reactive transport). Hence, the porosity value used for residence time calculation in the reactive transport models is assumed to be an average. This allows for both types of flow to be plausible, whether in a highly porous medium or large channels surrounded by less porous rock.

\bsk

The Indian Summer Monsoon (ISM) in Nepal is characterised by a strong seasonal reversal of winds, which brings heavy rainfall to the region during the summer months, and dry conditions during the winter (Bookhagen and Burbank, 2010). The monsoon brings a large amount of precipitation to the region. Oxygen isotope measurements suggest most of the precipitation occurs in the higher elevation parts of the catchment, and this is supported by remotely sensed rainfall estimates in the region (Acharya et al, 2020; Bookhagen and Burbank, 2010). Precipitation and discharge relationships in the Himalayas have been used to suggest that there is a three month lag in the response of the river to precipitation (Andermann et al., 2012). The residence time of groundwater can be used to quantify this delay and nature of its origin, given that rain is the main source of recharge to the groundwater system (Illien et al, 2021). Seasonal variation in rainfall is thought to relate to different hydrological regimes, whereby river discharge and precipitation are 'coupled' when there is a significant enough amount of water to recharge the groundwater system. (Illien et al, 2021) The seasonal variation in precipitation therefore also translates to a variation in runoff, whereby this is twelve times stronger during the monsoon than during the dry season (Sharma, 1997). 

\bsk

Changes in climate contribute to changes in the monsoonal system dynamics. The start of the monsoon has not changed in Nepal, but the end has been delayed. This has led to more intense precipitation on a per day basis, which is detrimental for crops in the winter season due to lack of moisture. Intense precipitation is also considered the main climatic cause of flooding (Panthi et al, 2015; Baniya et al, 2012). "One-off" landslide events transport as much as four times the flux of sediment deposited in the valley in a year (Chen C et al., 2023). These events are thought to be increasing in frequency over recent years as a result of climate change, increasing the erosion rate in these areas (Adhikari et al, 2023). In particular, effects of a flash flood in 2021 are still visible in the area, with damage done to several bridges and hundreds of families. 

\bsk

In this study, spring and rain samples from the Melamchi region of Nepal are used as a case study to investigate the weathering rates in a kinetically-limited catchment. (ref map) The sample dataset consists of 372 samples spanning four field campaigns over three years (2021-2024), as well as more recent year-long bi-weekly timeseries data from stream and spring samples in sites across the catchment. Of those, 68 were collected in September 2024 by a team with researchers from the University of Cambridge and Kathmandu University for this study. This dataset comprises major ion concentrations, alkalinity, and radiogenic strontium isotopes from the Melamchi catchment. 


% \newpage

% \section{Literature Review}

% \subsection{Defining Weathering}

% As water passes through the subsurface, it interacts with rock. This causes the addition of solute species to water, and the formation of stable secondary minerals through the dissolution of primary minerals formed at different pressure and temperature. Dissolved CO$_2$ derived from the atmosphere present in rainfall makes it slightly acidic:\\
% \begin{equation}
% H_2O + CO_2 \rightarrow H_2CO_3 \rightarrow HCO_3^- + H^+
% \end{equation}\\
% Once the rainfall enters the soil as groundwater, the acidicity is further increased by the presence of decomposition of organic matter, and CO$_2$ production from organic activity in the soil.

% \bsk

% The rate of weathering is dependent on the mineralogy of the rock. Different minerals weather at different rates: quartz > albite > mafic silicates > anorthite > carbonates. Therefore, the most reactive minerals will contribute disprportionately to the solute load of the water (Shand et al, 1999). In the Melamchi catchment, only the weathering of carbonic acid is considered. Sulfuric acid is also often considered a big player in weathering, but its impact is not considered in this study because the pyrite deposits required for its formation are not present in lithological studies of the Melamchi region.

% \newpage

% \begin{tcolorbox}[
%     colback=customcolor, % Use the defined custom color for background
%     colframe=white,      % Set the frame color to white (invisible)
%     sharp corners,       % Straight edges
%     boxrule=0pt,         % No border width
%     breakable,           % Allow the box to break into multiple pages (if needed)
%     width=\dimexpr\textwidth+2cm\relax, % Make the box wider than \textwidth
%     enlarge left by=-1cm,   % Shift box left to center the extra width
%     leftrule=0mm,        % No left border
%     rightrule=0mm,       % No right border
%     toprule=0mm,         % No top border
%     bottomrule=0mm       % No bottom border
% ]
% \textbf{\Large Box 1: Chemical Weathering Reactions}
% \vspace{-3mm}
% \myline\\
% {\footnotesize
% Chemical weathering reactions are dependent on the lithology and acid content of the water. Below are the primary reactions that characterise carbonic acid weathering worldwide.

% \bsk

% \textbf{Carbonic Acid Weathering of Carbonate}\\
% This reaction has no effect on atmospheric CO$_2$ levels in the long term.

%     \begin{center}
    
%     Short Term (10$^3$ years):
%     \begin{equation}
%     CaCO_3 + CO_2 + H_2O \rightarrow Ca^{2+} + 2HCO_3^-
%     \end{equation}
    
%     Long Term (10$^6$ years):
%     \begin{equation}
%     Ca^{2+} + 2HCO_3^- \rightarrow CaCO_3 + CO_2 + H_2O
%     \end{equation}

%     \end{center}
    
    
% \textbf{Carbonic Acid Weathering of Silicate}\\ This produces net CO$\ttmath{_2}$ drawdown. Reactions are written for a generic Ca-rich plagioclase mineral.

%     \begin{center}

%     Short Term (10$^3$ years):
%     \begin{equation}
%     2CO_2 + 3H_2O + CaAl_2Si_2O_8 \rightarrow Ca^{2+} + 2HCO_3^- + Al_2Si_2O_5(OH)_4
%     \end{equation}
    
%     Long Term (10$^7$ years):
%     \begin{equation}
%     \ttmath{Ca^{2+} + 2HCO_3^- \rightarrow CaCO_3 + CO_2 + H_2O}
%     \end{equation}
    
%     \end{center}

% }
% \end{tcolorbox}


% \subsection{Geological Controls on Weathering}


% The rate of weathering is dependent on the mineralogy of the rock. Different minerals weather at different rates: quartz > albite > mafic silicates > anorthite > carbonates. Therefore, the most reactive minerals will contribute disprportionately to the solute load of the water (Shand et al, 1999). In the Melamchi catchment, only the weathering of carbonic acid is considered. Weathering of sulfide minearls is also a big player in the global weathering budget, but its impact is not considered in this study because the marine deposits required for its formation are not present in the Melamchi region (Bufe et al., 2021). Differences in lithology are thought to affect weathering. Geological differences lead to differences in soil composition, landscape features, vegetation, and climate which in turn affect the rates of reaction. Logically, the contribution of one lithology to weathering is correlated to its spatial extent in the catchment (Stallard and Edmond, 1983). Porosities vary widely across a catchment depending on the rock type encountered (Singh et al, 1987; David et al, 1994). Porosity also increases as a rock becomes more weathered (Marques et al, 2009). Weathering regimes can be classified as either transport-limited or kinetically limited (West et al., 2005). West et al. distinguish the two regimes by the rate of erosion in the catchment. In low erosion rate settings, weathering is transport-limited due to limited mineral supply. Weathering here is therefore proportional to the material eroded. In high erosion rate settings, weathering is kinetically-limited due to an abundant mineral supply. Rapidly eroding catchments like Melamchi are therefore likely kinetically-limited. Soil properties and topography are used to identify different "weathering regimes" in the subsurface (Pedrazas et al, 2021). Indeed, bedrock strength is thought to be more dependent on weathering than mineral or textural differences between the metamorphic lithologies in the Himalayas (Medwedeff et al, 2021).



% \subsection{Reaction Rates in Natural and Laboratory Settings}

% Rates of reaction during weathering comprise both dissolution and precipitation, and chemical equilibrium is defined as that state where these are balanced and equal. Rates of reaction are thought to be different depending on whether they are measured in the field or in a laboratory (Maher et al., 2009). This difference has been explained by denoting 'extrinsic' qualities that are variable in the field, such as permeability and mineral/fluid ratios (White and Brantley, 2003). The rate of reaction of a system has also been linked to the free energy of the system, with laboratory rates being calculated significantly further away from equilibrium than field rates (Kampman et al, 2009). This implies that field localities are closer to equilibrium than laboratory-derived rates might suggest. 



% \subsection{Response to Monsoonal Precipitation}

% The Indian Summer Monsoon (ISM) is characterised by a strong seasonal reversal of winds, which brings heavy rainfall to the region during the summer months, and dry conditions during the winter (Bookhagen and Burbank, 2010). The monsoon brings a large amount of precipitation to the region. Oxygen isotopes suggest most of the precipitation occurs in the higher elevation parts of the catchment, and this is supported by remotely sensed rainfall estimates in the region (Acharya et al, 2020; Bookhagen and Burbank, 2010). Studying small catchments gives the opportunity to attribute large changes in water chemistry to seasonal climate changes like the monsoon (Tipper et al., 2006).  Seasonal variation in rainfall is thought to relate to different hydrological regimes, whereby river discharge and precipitation are 'coupled' when there is a significant enough amount of water to recharge the groundwater system. (Illien et al, 2021) The seasonal variation in precipitation therefore also translates to a variation in runoff, whereby this is twelve times stronger during the monsoon than during the dry season (Sharma, 1997). 

% \bsk

% Andermann et al. (2012) report anticlockwise hysteresis loops of precipitation against discharge (include figure?), and suggest that there is a three month lag in the response of the river to precipitation.  The delay in river discharge is a topic of debate. Andermann et al. (2012) propose that at lower elevations, it is more likely due to groundwater storage of precipitation in the fractured basement. Bookhagen and Burbank (2010) suggest that the delay of precipitation and discharge may be due to the response of glaciers at higher elevations. They also suggest evaportranspiration has a low impact on the hydrological budget of the Himalayas, less than 10\%. Other studies have found that catchments with little glacial input show the same delay, suggesting that the former hypothesis may be more relevant to this discussion (McGuire et al, 2005). The residence time of groundwater can hence be used to quantify this delay and nature of its origin, given that rain is the main source of recharge to the groundwater system (Illien et al, 2021). 


% \subsection{Effects of a Changing Climate}

% Changes in climate contribute to changes in the monsoonal system dynamics. The start of the monsoon has not changed in Nepal, but the end has been delayed. This has led to more intense precipitation on a per day basis, which is detrimental for crops in the winter season due to lack of moisture. Intense precipitation is also considered the main climatic cause of flooding (Panthi et al, 2015; Baniya et al, 2012). 

% \bsk

% "One-off" landslide events transport as much as four times the flux of sediment deposited in the valley in a year (Chen C et al., 2023). These events are thought to be increasing in frequency over recent years as a result of climate change, increasing the erosion rate in these areas (Adhikari et al, 2023). In particular, effects of a flash flood in 2021 are still visible in the area, with damage done to several bridges and hundreds of families. 



% Summary of the key questions (commented):
% 1. What main challenge does this study address?
% 2. What prior work and prevailing perspectives exist?
% 3. What does this investigation aim to accomplish?
% 4. Which results are highlighted by this study?
% 5. How have previous models approached the problem?
% 6. In what ways does this work differ from existing studies?
% 7. Why is this approach potentially more beneficial?
% 8. Which specific questions does this study seek to answer?







