
\section{Materials and Methods}
% Discuss the analytical methods used in the study.
% - Refer to relevant data tables.



\subsection{Field Samples}
Two types of water body were sampled in the field: springs and rain. 
Both were measured for temperature, pH and TDS on a Hanna Instruments ****** and  and EXTECH DO700. 
The springs were measured in situ and the rain was measured before titrating, 24 hours within having been collected. 
Six aliquots were collected for each spring for anion, cation, titration, DIC, isotope and archive purposes respectively. 
Rain samples had a smaller yield and so only two aliquots were collected, for isotope and archive purposes. 
Both water bodies were filtered through ***** in a filtration unit prior to bottling. 
Cation and archive samples were acidified with concentrated HNO$_3$ to give a pH of ~2 (why? to keep the ions in solution). 
Titration samples were used with a Hach digital titrator with 0.05M HCl to calculate the alkalinity of the water according to the Gran Method.

\subsection{Discharge Estimates}

\subsection{Erosion Estimates}

\subsection{Evapotranspiration Estimates}

\subsection{Borehole Measurements}

\subsection{Water Ion Measurements}

\subsection{Water Isotope Measurements}

\subsection{Sediment Measurements}
