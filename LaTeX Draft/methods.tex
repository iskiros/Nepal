
\section{Data collection and analysis}
% Discuss the analytical methods used in the study.
% - Refer to relevant data tables.



\subsection{Field Sampling}
Both springs and rain were sampled in the field. Springs were sampled according to locations visited in past expeditions. Rain was collected in a rain gauge along several transects. Both water bodies were measured in the field for temperature, pH and TDS on a Hanna Instruments HI-991300 and and EXTECH DO700. Samples were also titrated using a Hach digital titrator with 0.0625M HCl to calculate the alkalinity of the water following the Gran Method (Gran, 1952). The field measurements were done 24 hours within having been collected.  Six aliquots were collected for each spring for anion, cation, titration, DIC, isotope and archive purposes respectively.  Rain samples had a smaller yield and so only three aliquots were collected, for ion, isotope and archive purposes. Both water body types were filtered through a 0.2µm PES membrane in a filtration unit prior to bottling. Cation and archive samples were acidified with concentrated HNO$_3$ to give a pH of $\sim$2, keeping the cations in solution. 



%Titration uncertainty!! Use the mean square error of the titration to calculate the uncertainty of the alkalinity.



\subsection{Major and Trace Element Analysis}

Cation concentrations were determined using a Agilent Technologies 5100 Inductively-Coupled Plasma Optical Emission Spectrometer (ICP-OES) using a calibration line made from a Nepalese spring stock solution.
Anion concentrations were determined using a Dionex ICS-5000 Ion Chromatograph against the Battle-02 standard calibration line. Associated 
uncertainties range between 5-10\% for cations and 10-15\% for anions.

%fact check the anion uncertainty

\subsection{Isotope Analysis}

Samples for radiogenic strontium analysis were dried down to provide at least 10 $\mu$g of Sr. Samples were then dissolved in aqua regia (3:1 HNO$_3$:HCl) to remove any additional organic matter. Once dried down again, they were added to 3 ml teflon columns with Eichrom SrSpec$^{\textcopyright}$ resin pipetted in. Once washing the column three times with Milli-Q$^{\textregistered}$ water, it was primed with 3M HNO$_3$. The sample was centrifuged then loaded onto the column avoiding any solids. The column was then washed a total of three times with 3M HNO$_3$ to remove cations. Lastly the column was eluted to a beaker with Milli-Q$^{\textregistered}$ water to collect the Sr. Once dried, the samples were dissolved in 3N HNO$_3$, centrifuged and then diluted for analysis on a Thermo Scientific Neptune Plus MC-ICP-MS. \textcolor{red}{Write about uncertainties from machine}

\bsk

Samples were also analysed for stable oxygen and deuterium isotopes on a Picarro L2140-i portable analyser, using cavity ring-down spectroscopy, with an average precision of 0.05, 0.09 and 0.57 \textperthousand\ for $\delta^{17}O$, $\delta^{18}O$ and $\delta D$ respectively. 


\subsection{Estimates of Uncertainty}

Uncertainties were propagated using a Monte Carlo method. This leverages randomness to calculate uncertainties. Randomly adding noise to measured data within an estimated uncertainty range, the simulation shows how adding that noise propagates through the calculations. \textcolor{red}{Better definition here. Uncertainty table needed}

\newpage


\section{Cyclic and Hydrothermal Correction}

Rain input is a significant factor in the chemical composition of groundwater and rivers (Drever, 1997).\textcolor{red}{Most of the chloride is thought to come from the rain...} Spring water is corrected for rain input according to the average concentration for the closest rain sample collected in this field season. Whilst it would be best to perform a temporal correction, a spatial correction where the rain is collected from the same catchment as the analysed spring samples is already a significant improvement over the current literature (Bickle, 2015 AJS).

\bsk

To remove the contribution of the rain the following formula is used for any element X:

\begin{center}
{\Large
$[X]_{rain-corrected}$  = $[X]_{river}$ - $(Cl_{river} - Cl^*_{river})\ddfrac{[X]_{rain}}{[Cl]_{rain}}$}

\end{center}

Where $[Cl]^*_{river}$ is is calculated by subtracting the concentration of chloride in the rain from that in the river (Tipper et al, 2006).
$Cl^{*}$ is taken to be zero if the concentration of chloride in the rain is greater than concentration of river. Evapotranspiration is not considered by this model, and
studies which show that it plays a minor role, accounting for less than 10\% of the hydrological budget in the Himalayas (Andermann et al (2012); Bookhagen and Burbank, 2010).




% {\tiny
% This formula is used because it allows for evapotranspiration to be corrected for in a later revision. 

% Evapotranspiration (ET henceforth) is a term that describes the sum of evaporation and plant transpiration 
% from the earth's land surface to atmosphere, including soil and vegetation.[Cite]. 
% ET is crop dependent and intuitively varies as rainfall with time. ET is calculated 
% using the Penman-Monteith method, and there exists literature which compile global 
% annual databases of ET. These can be compared for the Cañete to rainfall, and the above equation can be modified like so:

% \begin{center}
% {\Large
% $[X]_{rain-corrected}$ = $[X]_{river}$ - $\frac{[X]_{rain}}{[Cl]_{rain}} *[Cl]_{lowest-river}  * ETF   $}

% \end{center}

% Where ETF is an evapotranspiration factor relating the relative amonts of rain to evapotranspiration. 
% For example,  if half the rain gets evaporated, then the samples will be more concentrated when corrected (and the minimum ET is zero).
% }


In those cases where $Cl^*$ is not zero then, a primary rain correction is simply:


\begin{center}
{\Large
$[X]_{rain-corrected}$  = $[X]_{river}$ - $[X]_{rain}$}

\end{center}

\bsk

Once the samples have been corrected for rain input, the remaining $[Cl]^{-}$ is assumed to be derived from evaporites encountered in the flow path.

\bsk

% Do we do this?
Hence, the sample with the highest $[Cl]^{-}$ is used to correct the ions in a similar fashion to how the most dilute sample was used above:

\begin{center}
{\Large
$[X]_{evaporite-corrected}$  = $[X]_{rain-corrected}$ - $\frac{[X]}{[Cl]}_{highest-Cl} * [Cl]_{rain-corrected}$}
\end{center}

This ensures that all chloride in the corrected sample is removed. The correction uses ionic ratios from the most concentrated water source, which acts as a proxy for the sediment imparting its signature.  In this way, the correction does not affect samples which do not have high Cl (and hence do not have a large evaporite contribution), but does decrease the concentration of ions for those that do.

\bsk

In the following sections, samples are plotted with the evaporite correction applied where possible. Only in those cases where no chloride was measured is the evaporite correction not applied.

% As is seen in the following section (ref), many samples in Traverse 2 have a high Cl concentration, but this is less of a concern for the other traverses.

\bsk

% As a further detail when modeling a groundwater flow, 
% it is unlikely that the composition of Sample$_{highest-Cl}$ is just a product of evaporites. 
% The water will also flow over silicate and carbonate rocks. 
% Under the assumption that the evaporite in question is stoichiometric halite, 
% any additional sodium in the Na:Cl ratio of Sample$_{highest-Cl}$ comes from 
% silicate weathering (primary silicate weathering, if you will). 
% Quantifying the effect of this primary weathering is highly complicated 
% and a topic worthy of its own investigation (eg my part III project). 
% However, it can be to a first approximation factored out if all of the ratios 
% are normalised such that Na:Cl is equal to 1 (i.e. divide each ratio by Na:Cl for Sample$_{highest-Cl}$).

%\subsection{Evapotranspiration Estimates}
%???


