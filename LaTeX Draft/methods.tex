
\section{Data collection and analysis}
% Discuss the analytical methods used in the study.
% - Refer to relevant data tables.



\subsection{Field Sampling}
Two types of water body were sampled in the field: springs and rain. Springs were sampled from the closest identified source in
 the study area, and rain was collected in a rain gauge.
Both were measured in the field for temperature, pH and TDS on a Hanna Instruments HI-991300 and  and EXTECH DO700. The field measurements 
were done at the source for the springs, and back at base for the rain before titrating, 
24 hours within having been collected. 
Six aliquots were collected for each spring for anion, cation, titration, DIC, isotope and archive purposes respectively. 
Rain samples had a smaller yield and so only three aliquots were collected, for ion, isotope and archive purposes. 
Both water body types were filtered through a 0.2µm PES membrane in a filtration unit prior to bottling. 
Cation and archive samples were acidified with concentrated HNO$_3$ to give a pH of $\sim$2, keeping the cations in solution. 
Samples were titrated with a Hach digital titrator with 0.0625M HCl to calculate the alkalinity of the water following the Gran Method (Gran, 1952).



%Titration uncertainty!! Use the mean square error of the titration to calculate the uncertainty of the alkalinity.



\subsection{Major and Trace Element Analysis}

Ion concentrations were measured in Cambridge once back from the field. Cation concentrations were determined using a Agilent Technologies 
5100 Inductively-Coupled Plasma Optical Emission Spectrometer (ICP-OES) using a calibration line made from a Nepalese spring stock solution.
Anion concentrations were determined using a Dionex ICS-5000 Ion Chromatograph against the Battle-02 standard calibration line. Associated 
uncertainties range between 5-10\% for cations and 10-15\% for anions.

%fact check the anion uncertainty

\subsection{Sr Isotope Analysis}

Samples were dried down to provide at least 1 $\mu$g of Sr. Samples were then dissolved in aqua regia (3:1 HNO$_3$:HCl) to remove any additional organic matter. Once dried down again, they were added to 30 $\mu$l teflon columns with Eichrom SrSpec$^{\textcopyright}$ resin pipetted in. Once washing the column three times with Milli-Q$^{\textregistered}$ water, it was primed with 3N HNO$_3$. The sample was centrifuged then loaded onto the column avoiding any solids. The column was then washed a total of three times with 3N HNO$_3$ to remove cations. Lastly the column was eluted to a beaker with Milli-Q$^{\textregistered}$ water to collect the Sr. Once dried, the samples were ...





\subsection{Cyclic and hydrothermal Correction + radiogenic?}

Rain input is a significant factor in the chemical composition of rivers (Drever, 1997).
Spring water is corrected for rain input according to the average concentration for the closest 
rain sample collected in this field season. % can remove if too much text

\bsk

To remove the contribution of the rain the following formula is used for any element X:

\begin{center}
{\Large
$[X]_{rain-corrected}$  = $[X]_{river}$ - $(Cl_{river} - Cl^*_{river})\ddfrac{[X]_{rain}}{[Cl]_{rain}}$}

\end{center}

Where $[Cl]^*_{river}$ is is calculated by subtracting the concentration of chloride in the rain from that in the river (Tipper et al, 2006).
$Cl^{*}$ is taken to be zero if the concentration of chloride in the rain is greater than concentration of river. Evapotranspiration is not considered by this model, because of
studies like Andermann et al (2012) which show that it plays a minor role, accounting for less than 10\% of the hydrological budget in the Himalayas.
They agree with Bookhagen and Burbank


%do we keep this??

\bsk
\bsk

% {\tiny
% This formula is used because it allows for evapotranspiration to be corrected for in a later revision. 

% Evapotranspiration (ET henceforth) is a term that describes the sum of evaporation and plant transpiration 
% from the earth's land surface to atmosphere, including soil and vegetation.[Cite]. 
% ET is crop dependent and intuitively varies as rainfall with time. ET is calculated 
% using the Penman-Monteith method, and there exists literature which compile global 
% annual databases of ET. These can be compared for the Cañete to rainfall, and the above equation can be modified like so:

% \begin{center}
% {\Large
% $[X]_{rain-corrected}$ = $[X]_{river}$ - $\frac{[X]_{rain}}{[Cl]_{rain}} *[Cl]_{lowest-river}  * ETF   $}

% \end{center}

% Where ETF is an evapotranspiration factor relating the relative amonts of rain to evapotranspiration. 
% For example,  if half the rain gets evaporated, then the samples will be more concentrated when corrected (and the minimum ET is zero).
% }
\bsk
\bsk


In those cases where $Cl^*$ is not zero then, a primary rain correction is simply:


\begin{center}
{\Large
$[X]_{rain-corrected}$  = $[X]_{river}$ - $[X]_{rain}$}

\end{center}

\bsk

Once the samples have been corrected for rain input, the remaining $[Cl]^{-}$ is assumed to be derived from evaporites encountered in the flow path. 

% Do we do this?
Hence, the sample with the highest $[Cl]^{-}$ is used to correct the ions in a similar fashion to how the most dilute sample was used above:

\begin{center}
{\Large
$[X]_{evaporite-corrected}$  = $[X]_{rain-corrected}$ - $\frac{[X]}{[Cl]}_{highest-Cl} * [Cl]_{rain-corrected}$}
\end{center}

This ensures that all chloride in the corrected sample is removed. 
The correction uses ionic ratios from the most concentrated water source, 
which acts as a proxy for the sediment imparting its signature. 
In this way, the correction does not affect samples which do not have high Cl 
(and hence do not have a large evaporite contribution), but does decrease the concentration of ions for those that do.

\bsk

% As a further detail when modeling a groundwater flow, 
% it is unlikely that the composition of Sample$_{highest-Cl}$ is just a product of evaporites. 
% The water will also flow over silicate and carbonate rocks. 
% Under the assumption that the evaporite in question is stoichiometric halite, 
% any additional sodium in the Na:Cl ratio of Sample$_{highest-Cl}$ comes from 
% silicate weathering (primary silicate weathering, if you will). 
% Quantifying the effect of this primary weathering is highly complicated 
% and a topic worthy of its own investigation (eg my part III project). 
% However, it can be to a first approximation factored out if all of the ratios 
% are normalised such that Na:Cl is equal to 1 (i.e. divide each ratio by Na:Cl for Sample$_{highest-Cl}$).

%\subsection{Evapotranspiration Estimates}
%???


