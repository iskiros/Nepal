\section{Appendix 1: Strontium Isotopes and their Applications}
Need to include a blurb about how they originate and what they mean.

\bsk

Following Faure (2001 - Origin of Igneous Rocks), the Sr concentrations of a two-component mixture can be expressed as:\\

\begin{equation}
    [Sr]_M = [Sr]_A f_A + [Sr]_B(1-f_A)
\end{equation}\\

These are related to the isotopic composition with:\\

\begin{equation}
    \left(\frac{^{87}Sr}{^{86}Sr}\right)_M = \left(\frac{^{87}Sr}{^{86}Sr}\right)_A f_A \frac{[Sr]_A}{[Sr]_M} + \left(\frac{^{87}Sr}{^{86}Sr}\right)_B(1-f_A)\frac{[Sr]_B}{[Sr]_M} 
\end{equation}\\

These equations can be combined, yielding:\\

\begin{equation}
    \left(\frac{^{87}Sr}{^{86}Sr}\right)_M = \frac{a}{[Sr]_M} + b; \quad \text{with:}
\end{equation}\\
\begin{equation}
    a = \frac{[Sr]_A [Sr]_B \left[ \left( \ddfrac{^{87}Sr}{^{86}Sr} \right)_B - \left( \ddfrac{^{87}Sr}{^{86}Sr} \right)_A \right]}{[Sr]_A - [Sr]_B}
\end{equation}\\
\begin{equation}
    b = \frac{[Sr]_A \left( \ddfrac{^{87}Sr}{^{86}Sr} \right)_A - [Sr]_B \left( \ddfrac{^{87}Sr}{^{86}Sr} \right)_B}{[Sr]_A - [Sr]_B}
\end{equation}\\

Plots of $\ddfrac{^{87}Sr}{^{86}Sr}$ against $\ddfrac{1}{Sr}$ that yield straight lines are therefore indicative of mixing trends. 



\section{Appendix 2: Derivation of reactive transport Models}


\begin{tcolorbox}[
    colback=white,
    colframe=white,
    sharp corners,
    boxrule=0pt,
    breakable,
    width=\dimexpr\textwidth+1cm\relax,
    enlarge left by=-0.5cm,
    leftrule=0mm, rightrule=0mm, toprule=0mm, bottomrule=0mm
]


\subsection{Fontorbe et al. (2013) - Null Hypothesis - Model Derivation}
\vspace{-5mm}
{\footnotesize

\noindent\rule{\textwidth}{0.5pt}

This model investigates silicon isotopic composition in the Ganges River, assuming constant reaction rates along flow paths.

\begin{multicols}{2}

    The first-order differential equation governing transport and reaction is given as:
    \vspace{10mm}
    \columnbreak
    \begin{equation}
    \phi \frac{\partial C}{\partial t} = -\omega \phi \frac{\partial C}{\partial z} + R_n(1-f)
    \end{equation} 
    
\end{multicols}

% \begin{table}[H]
%     \centering
%     {\footnotesize
%     \begin{tabular}{|c|c|c|}
%         \hline
%         \textbf{Variable} & \textbf{Definition} & \textbf{Units} \\
%         \hline
%         \(C\) & Concentration in water & \(\mu\)mol/L \\
%         \(\phi\) & Porosity of the medium & - \\
%         \(\omega\) & Fluid velocity along the flow path & m/s \\
%         \(z\) & Distance along the 1D flow path & m \\
%         \(R_n\) & Rate of reaction (dissolution) & mol/m\(^3\)/s \\
%         \(f\) & Fraction of Si reprecipitated in clay minerals & - \\
%         \hline
%     \end{tabular}
%     }
%     %\caption{Key variables and their definitions.}
%     \label{tab:variables}
% \end{table}  
    
\begin{multicols}{2}

To simplify, we introduce non-dimensional variables:
\vspace{10mm}
\columnbreak % --- break3 --- Move to the next column for transformation equations
\begin{equation}
    C' = \frac{C}{C_o},\quad z' = \frac{z}{h},\quad t' = \frac{t\omega}{h}
\end{equation}
    
\end{multicols}

\bsk

\begin{multicols}{2}% --- break4 --- Back to left for explaining transformation
    
    Rewriting Equation (1) using these new variables:
    \vspace{10mm}
    \columnbreak % --- break5 --- Equation for transformed form
    \begin{equation}
        \frac{\partial C'}{\partial t'} = -\frac{\partial C'}{\partial z'} + N_D(1-f)
    \end{equation}
    
\end{multicols}

\bsk

\begin{multicols}{2} % --- break6 --- Explain Damköhler number
    
    The Damköhler number (\(N_D\)) describes the relative importance of kinetic vs transport-controlled settings (Bethke, 2008):
    \vspace{10mm}
    \columnbreak
    \begin{equation}
        N_D = \frac{R_n h}{\phi C_o \omega}
    \end{equation}
    
\end{multicols}

\bsk

\begin{multicols}{2} % --- break7 --- Explain steady-state assumption
    
    Assuming a quasi-stationary state (\(\partial C'/\partial t' = 0\)), we get:
    \vspace{10mm}
    \columnbreak % --- break8 --- Equation for steady-state solution
    \begin{equation}
        C' = 1 + z'N_D(1-f)
    \end{equation}
    
\end{multicols}

\bsk

\begin{multicols}{2} % --- break9 --- Interpretation at flow path end
    
    At the end of the flow path (\(z = h, z' = 1\)), this simplifies to:
    \vspace{10mm}
    \columnbreak
    \begin{equation}
        N_D = \frac{C'_h - 1}{1-f}
    \end{equation}
    
\end{multicols}

\bsk

\begin{multicols}{2} % --- break10 --- Explain residence time
    
    The residence time of water along the flow path is:
    \vspace{10mm}
    \columnbreak
    \begin{equation}
        T_f = \frac{h}{\omega}
    \end{equation}
    
\end{multicols}

\begin{multicols}{2} % --- break11 --- Final set of equations
    
    At the end of the flow path:
    \vspace{10mm}


    \columnbreak
    \begin{equation}
        \frac{R_n h}{\phi C_o \omega} = \frac{C'_h - 1}{1-f}
    \end{equation}
    \begin{equation}
        \frac{R_n T_f}{\phi C_o} = \frac{C'_h - 1}{1-f}
    \end{equation}
    
\end{multicols}

\bsk

\begin{multicols}{2} % --- break12 --- Solve for residence time and reaction rate
    
    Solving for residence time \(T_f\) and reaction rate \(R_n\):
    \vspace{10mm}
    \columnbreak
    \begin{equation}
        T_f = \frac{(C_h - C_o)\phi}{(1-f)R_n}
    \end{equation}
    \begin{equation}
        R_n = \frac{(C_h - C_o)\phi}{(1-f)T_f}
    \end{equation}
    
\end{multicols}
% --- break13 --- Final notes
    
$C_h$ is the concentration at the end of the flow path, which is taken to be equal to the concentration of each spring $C$, assuming each spring represents a unique flow path. Convert \(T_f\) from \(10^{-9}\) seconds to years for practical use. 

\begin{table}[H]
    \centering
    %\renewcommand{\arraystretch}{1.3} % Adjust row height
    \begin{tabular}{|c|c|c|c|}
        \hline
        \multicolumn{4}{|c|}{\textbf{Fontorbe}} \\  
        \hline
        \textbf{Parameter} & \textbf{Definition} & \textbf{Units} & \textbf{Formula (Value)} \\  
        \hline
        $\phi$ & Porosity & - & - \\
        $\omega$ & Fluid velocity & m/s & - \\
        $h$ & Length of flow path & m & Variable \\
        $C_h$ & Concentration \@ end of flow path & $\mu$mol/L & Variable \\
        $C_0$ & Initial concentration & $\mu$mol/L & Rain Conc \\
        $f$ & Fraction reprecipitated & - & Order 0.5 \\
        $N_D$ & Non-dimensional number & - & $N_D = \frac{R_n h}{\phi C_0 \omega}$ \\
        $T_f$ & Residence time & $10^{-9}$ s & $T_f = \frac{h}{\omega}$ \\
        $R_n$ & Reaction rate & mol/m$^3$/s & $k\cdot S \cdot \rho \cdot 1000 \cdot X  $ \\
        $k$ & Reaction rate constant & mol/m$^2$/s & - \\
        $S$ & Specific surface area & m$^2$/g & - \\
        $\rho$ & Mineral density & kg/m$^3$ &  \\
        $X$ & Volume fraction of mineral in rock & $g_{min}/g_{rock}$ & 0.2 \\
        \hline
    \end{tabular}
    \caption{Key parameters and definitions for the Fontorbe model.}
    \label{tab:parameters1}
\end{table}
    
}

\end{tcolorbox}


\FloatBarrier

\newpage
\begin{tcolorbox}[
    colback=white,
    colframe=white,
    sharp corners,
    boxrule=0pt,
    breakable,
    width=\dimexpr\textwidth+1cm\relax,
    enlarge left by=-0.5cm,
    leftrule=0mm, rightrule=0mm, toprule=0mm, bottomrule=0mm
]


\subsection{Maher Model Derivation}
\vspace{-5mm}
{\footnotesize

\noindent\rule{\textwidth}{0.5pt}

This model describes a reaction-based approach to solute transport, following Maher and Chamberlain (2013, 2014).

\bsk

The first-order differential equation governing transport and reaction is:\\

\begin{equation}
\frac{dc}{dt} = -\frac{q}{\theta} \frac{dc}{dx} + \sum_{i} \mu_i R_{d,i} \left( 1 - \left( \frac{c}{c_{\text{eq}}} \right)^{n_i} \right)^{m_i} - \sum_{i} \mu_i R_{p,i} \left( 1 - \left( \frac{c}{c_{\text{eq}}} \right)^{n_i} \right)^{m_i}
\end{equation}\\


Where c is the concentration, q is the fluid flux, theta is the volumetric water content, x is the position along the flow path, mu is the stoichiomeric coefficient, R is the rate of reaction for dissolution and precipitation respectively, c$_eq$ is the equilibrium concentration, and n and m are non-linear parameters (Maher and Chamberlain, 2013). 

\begin{equation}
\frac{dc}{dx} =  \frac{q}{\theta}\sum_{i} \mu_i R_{d,i} \left( 1 - \left( \frac{c}{c_{\text{eq}}} \right)^{n_i} \right)^{m_i} - \frac{q}{\phi}\sum_{i} \mu_i R_{p,i} \left( 1 - \left( \frac{c}{c_{\text{eq}}} \right)^{n_i} \right)^{m_i}
\end{equation}\\

\begin{multicols}{2}

Defining the net reaction rate:

\columnbreak

\begin{equation}
R_n = \sum_{i} \mu_i R_{d,i} - \sum_{i} \mu_i R_{p,i}
\end{equation}

\end{multicols}

\begin{multicols}{2}

Maher and Chamberlain describe that the reaction rate decreases linearly with approach to equilibrium.

\columnbreak
\begin{equation}
\frac{dc}{dt} = R_n \left( 1 - \frac{c}{c_{\text{eq}}} \right)
\end{equation}

\end{multicols}

Solving for \( c(x) \), following Maher and Chamberlain (2013):

\begin{equation}
c(x) = c_0 \exp\left(-\frac{R_n \theta x}{q c_{\text{eq}}} \right) + c_{\text{eq}} \left( 1 - \exp\left(-\frac{R_n \theta x}{q c_{\text{eq}}} \right) \right)
\end{equation}\\


\begin{multicols}{2}

We also define the residence time \( T_f \):

\columnbreak

\begin{equation}
T_f = \frac{L\phi}{q}
\end{equation}

\end{multicols}

Thus, at residence time \( T_f \):

\begin{equation}
c(T_f) = c_0 \exp\left(-\frac{R_n T_f}{c_{\text{eq}}} \right) + c_{\text{eq}} \left( 1 - \exp\left(-\frac{R_n T_f}{c_{\text{eq}}} \right) \right)
\end{equation}


\begin{multicols}{2}

Following Maher and Chamberlain (2014), this can be rewritten as:
\columnbreak
\begin{equation}
    C =  \frac{C_{\text{0}}}{1 + \tau D_w / q} + C_{\text{eq}} \frac{\tau D_w / q}{1 + \tau D_w / q}
\end{equation}
    
\end{multicols}

\begin{multicols}{2}

where:

\columnbreak
\begin{equation}
\tau = e^2; \quad D_w = \frac{L\theta R_n}{C_{\text{eq}}}
\end{equation}

\end{multicols}

\begin{multicols}{2}

Rewriting the equation:

\columnbreak

\begin{equation}
    C = \frac{C_{\text{0}} + C_{\text{eq}} \cdot T_f\left(e^2 \cdot R_n / C_{\text{eq}}\right)}{1 + T_f\left(e^2 \cdot R_n / C_{\text{eq}}\right)}
\end{equation}

\end{multicols}

\begin{multicols}{2}

Solving for residence time \( T_f \), and reaction rate \( R_n \) which can be compared to the Fontorbe model:
\columnbreak
\begin{equation}
    T_f = \frac{C_{eq} \cdot \left(C - C_0\right)}{e^2 R_n \left( C_{\text{eq}} - C \right)}
\end{equation}
\begin{equation}
    R_n = \frac{C_{eq} \cdot \left(C - C_0\right)}{e^2 T_f \left( C_{\text{eq}} - C \right)}
\end{equation}

\end{multicols}

Note: Convert \( T_f \) from \( 10^{-3} \)s to years for practical use.\\

\begin{table}[H]
    \centering
    %\renewcommand{\arraystretch}{1.3} % Adjust row height
    \begin{tabular}{|c|c|c|c|}
        \hline  % DOUBLE BOLD LINE
        \multicolumn{4}{|c|}{\textbf{Maher Model Parameters}} \\  
        \hline
        \textbf{Parameter} & \textbf{Definition} & \textbf{Units} & \textbf{Formula (Value)} \\  
        $L$ & Length of flow path & m & Variable \\
        $q$ & Flow rate & m/s & Variable \\
        $\phi$ & Porosity & - & 0.3 (but variable) \\
        $\theta$ & Volumetric water content & - & Variable \\
        $R_n$ & Net reaction rate & mol/L/s & $\rho_{sf} \cdot k \cdot A \cdot X_r $ \\
        $\rho_{sf}$ & Mass mineral / Fluid Volume ratio & g/L & $1000 \cdot \rho_b / \phi$ \\
        $\rho_b$ & Plagioclase density & g/cm$^3$ & - \\
        $k$ & Reaction rate constant & mol/m$^2$/s & - \\
        $A$ & Specific surface area & m$^2$/g & 0.1-1 \\
        $X_r$ & Mineral concentration in fresh rock & $g_{min}/g_{rock}$& Wt\% in rock \\
        $\tau$ & Scaling factor & - & $\tau = e^2$ \\
        $D_w$ & Damkohler Coefficient & m$^2$/s & $D_w = \frac{L \phi R_n}{C_{\text{eq}}}$ \\
        $T_f$ & Residence time & $10^{-6}$ s & $T_f = \frac{L \phi}{q}$ \\
        $C_{eq}$ & Equilibrium concentration & $\mu$mol/L & Max Catchment \\
        $C_0$ & Initial concentration & $\mu$mol/L & Rain Conc \\
        \hline
    \end{tabular}
    \caption{Key parameters and definitions for the Maher model.}
    \label{tab:parameters2}
\end{table}

\FloatBarrier

}

\end{tcolorbox}

