

\section{Appendix 1: Dimensional Analysis of \(T_f\)}

\subsection*{Fontorbe Model}

The residence time \(T_f\) is given by:

\begin{equation}
    T_f  = \frac{\left(C_h - C_0\right)\cdot\phi}{\left(1-f\right)\cdot R_n}
\end{equation}

where the parameters have the following units:

\begin{itemize}
    \item \( C_h, C_0 \) (Concentration): \( \mu\text{mol}/\text{L} \)
    \[
    1~\mu\text{mol}/\text{L} = \frac{10^{-6}~\text{mol}}{10^{-3}~\text{m}^3} = 10^{-3}~\frac{\text{mol}}{\text{m}^3}
    \]
    \item \( \phi \) (Porosity) is dimensionless.
    \item \( f \) (Fraction reprecipitated) is dimensionless.
    \item \( R_n \) (Reaction rate) is given by:
    \begin{equation}
        R_n = k \cdot S \cdot \rho \cdot 10^3 \cdot X \cdot (1-\phi)
    \end{equation}
    where:
    \begin{itemize}
        \item \( k \) (Reaction rate constant): \( \left[\frac{\text{mol}}{\text{m}^2 \cdot \text{s}}\right] \)
        \item \( S \) (Specific surface area): \( \left[\frac{\text{m}^2}{\text{g}}\right] \)
        \item \( \rho \) (Mineral density): \( \left[\frac{\text{kg}}{\text{m}^3}\right] \) \\
        Since \( 1~\text{kg} = 10^3~\text{g} \), we write:
        \[
        \rho = \frac{10^3~\text{g}}{\text{m}^3}
        \]
        \item \( X \) (Volume fraction of mineral in rock) is dimensionless.
        \item \( (1 - \phi) \) is dimensionless.
    \end{itemize}
\end{itemize}

\subsubsection*{Step-by-Step Dimensional Analysis of \(R_n\)}

First, multiplying \( k \) and \( S \):

\[
k \cdot S = \left(\frac{\text{mol}}{\text{m}^2 \cdot \text{s}}\right) \times \left(\frac{\text{m}^2}{\text{g}}\right) = \frac{\text{mol}}{\text{g} \cdot \text{s}}
\]

Next, multiplying by \( \rho \):

\[
\left(\frac{\text{mol}}{\text{g} \cdot \text{s}}\right) \times \left(\frac{10^3 \text{g}}{\text{m}^3}\right) = \frac{10^3 \text{mol}}{\text{m}^3 \cdot \text{s}}
\]

Since \( X \) and \( (1 - \phi) \) are dimensionless, they do not affect the units.

Thus, the final unit of \( R_n \) is:

\[
R_n = \frac{10^3 \text{mol}}{\text{m}^3 \cdot \text{s}}
\]

\subsubsection*{Final Dimensional Analysis of \(T_f\)}

Substituting the units:

\[
T_f = \frac{(C_h - C_0) \cdot \phi}{(1 - f) \cdot R_n}
\]

Since \( C_h - C_0 \) has units of:

\[
10^{-3} \frac{\text{mol}}{\text{m}^3}
\]

and \( R_n \) has units of:

\[
\frac{10^3 \text{mol}}{\text{m}^3 \cdot \text{s}},
\]

we get:

\[
T_f = \frac{10^{-3} \frac{\text{mol}}{\text{m}^3}}{10^3 \frac{\text{mol}}{\text{m}^3 \cdot \text{s}}}
\]

Canceling \( \frac{\text{mol}}{\text{m}^3} \):

\[
T_f = \frac{10^{-3}}{10^3} \cdot s = 10^{-6} s
\]

\subsubsection*{Conclusion}

The predicted unit of \( T_f \) is:

\[
T_f \sim 10^{-6} \text{ s}
\]

\newpage


\subsection*{Maher Model}

\subsubsection*{Given Equation for \( T_f \)}

The residence time \( T_f \) is given by:


\begin{equation}
    T_f = \frac{C_{eq} \cdot \left(C - C_0\right)}{e^2 R_n \left( C_{\text{eq}} - C \right)}
\end{equation}\\


where the parameters have the following units:

\begin{itemize}
    \item \( C_0, C, C_{\text{eq}} \) (Concentrations) in \( \mu \)mol/L:
    \[
    1~\mu\text{mol}/\text{L} = 10^{-6}~\frac{\text{mol}}{\text{L}}
    \]
    Thus,
    \[
    C_0, C, C_{\text{eq}} \sim 10^{-6} \frac{\text{mol}}{\text{L}}
    \]
    \item \( R_n \) (Net reaction rate) is defined as:
    \[
    R_n = \rho_{sf} \cdot k \cdot A \cdot X_r
    \]
    where:
    \begin{itemize}
        \item \( \rho_{sf} \) (Mass mineral/Fluid Volume ratio) has units:
        \[
        \rho_{sf} \sim \frac{\text{g}}{\text{L}}
        \]
        \item \( k \) (Reaction rate constant):
        \[
        k \sim \frac{\text{mol}}{\text{m}^2 \cdot \text{s}}
        \]
        \item \( A \) (Specific surface area):
        \[
        A \sim \frac{\text{m}^2}{\text{g}}
        \]
        \item \( X_r \) (Mineral concentration in fresh rock) is dimensionless.
    \end{itemize}
\end{itemize}

\subsubsection*{Step-by-Step Dimensional Analysis of \( R_n \)}

Expanding \( R_n \):

\[
R_n = \left(\frac{\text{g}}{\text{L}}\right) \times \left(\frac{\text{mol}}{\text{m}^2 \cdot \text{s}}\right) \times \left(\frac{\text{m}^2}{\text{g}}\right)
\]

Canceling \( \text{g} \) and \( \text{m}^2 \):

\[
R_n = \frac{\text{mol}}{\text{L} \cdot \text{s}}
\]

Thus, the reaction rate \( R_n \) has the units:

\[
R_n \sim \frac{\text{mol}}{\text{L} \cdot \text{s}}
\]

\subsubsection*{Dimensional Analysis of \( T_f \)}

Substituting the units into:


\begin{equation}
    T_f = \frac{C_{eq} \cdot \left(C - C_0\right)}{e^2 R_n \left( C_{\text{eq}} - C \right)}
\end{equation}\\


- **Numerator:**  
  \[
   C_{eq} \cdot \left(C - C_0\right) \sim 10^{-12} \frac{\text{mol}^2}{\text{L}^2}
  \]

- **Denominator:**  
  \[
  e^2 R_n \left( C_{\text{eq}} - C \right) 
  \]

  Since \( e^2 \) is **dimensionless**, we are left with:

  \[
  e^2 R_n \sim 10^{-6} \frac{\text{mol}^2}{\text{L}^2 \cdot \text{s}}
  \]

Now, dividing:

\[
T_f = \frac{10^{-12} \frac{\text{mol}^2}{\text{L}^2}}{\frac{10^{-6}\text{mol}^2}{\text{L}^2 \cdot \text{s}}}
\]

Canceling \( \frac{\text{mol}}{\text{L}} \):

\[
T_f = 10^{-6} s
\]

\subsubsection*{Conclusion}

The predicted unit of \( T_f \) is:

\[
T_f \sim 10^{-6} \text{ s}
\]

which corresponds to:

\[
T_f = 1~\mu\text{s} \quad (\text{microseconds}).
\]


