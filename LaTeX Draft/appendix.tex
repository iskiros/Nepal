
\section{Appendix 1: Derivation of reactive transport Models}


\begin{tcolorbox}[
    colback=white,
    colframe=white,
    sharp corners,
    boxrule=0pt,
    breakable,
    width=\dimexpr\textwidth+1cm\relax,
    enlarge left by=-0.5cm,
    leftrule=0mm, rightrule=0mm, toprule=0mm, bottomrule=0mm
]


\subsection{Fontorbe et al. (2013) - Null Hypothesis - Model Derivation}
\vspace{-5mm}
{\footnotesize

\noindent\rule{\textwidth}{0.5pt}

This model investigates silicon isotopic composition in the Ganges River, assuming constant reaction rates along flow paths.

\begin{multicols}{2}

    The first-order differential equation governing transport and reaction is given as:
    \vspace{10mm}
    \columnbreak
    \begin{equation}
    \phi \frac{\partial C}{\partial t} = -\omega \phi \frac{\partial C}{\partial z} + R_n(1-f)
    \end{equation} 
    
\end{multicols}

% \begin{table}[H]
%     \centering
%     {\footnotesize
%     \begin{tabular}{|c|c|c|}
%         \hline
%         \textbf{Variable} & \textbf{Definition} & \textbf{Units} \\
%         \hline
%         \(C\) & Concentration in water & \(\mu\)mol/L \\
%         \(\phi\) & Porosity of the medium & - \\
%         \(\omega\) & Fluid velocity along the flow path & m/s \\
%         \(z\) & Distance along the 1D flow path & m \\
%         \(R_n\) & Rate of reaction (dissolution) & mol/m\(^3\)/s \\
%         \(f\) & Fraction of Si reprecipitated in clay minerals & - \\
%         \hline
%     \end{tabular}
%     }
%     %\caption{Key variables and their definitions.}
%     \label{tab:variables}
% \end{table}  
    
\begin{multicols}{2}

To simplify, we introduce non-dimensional variables:
\vspace{10mm}
\columnbreak % --- break3 --- Move to the next column for transformation equations
\begin{equation}
    C' = \frac{C}{C_o},\quad z' = \frac{z}{h},\quad t' = \frac{t\omega}{h}
\end{equation}
    
\end{multicols}

\bsk

\begin{multicols}{2}% --- break4 --- Back to left for explaining transformation
    
    Rewriting Equation (1) using these new variables:
    \vspace{10mm}
    \columnbreak % --- break5 --- Equation for transformed form
    \begin{equation}
        \frac{\partial C'}{\partial t'} = -\frac{\partial C'}{\partial z'} + N_D(1-f)
    \end{equation}
    
\end{multicols}

\bsk

\begin{multicols}{2} % --- break6 --- Explain Damköhler number
    
    The Damköhler number (\(N_D\)) describes the relative importance of kinetic vs transport-controlled settings (Bethke, 2008):
    \vspace{10mm}
    \columnbreak
    \begin{equation}
        N_D = \frac{R_n h}{\phi C_o \omega}
    \end{equation}
    
\end{multicols}

\bsk

\begin{multicols}{2} % --- break7 --- Explain steady-state assumption
    
    Assuming a quasi-stationary state (\(\partial C'/\partial t' = 0\)), we get:
    \vspace{10mm}
    \columnbreak % --- break8 --- Equation for steady-state solution
    \begin{equation}
        C' = 1 + z'N_D(1-f)
    \end{equation}
    
\end{multicols}

\bsk

\begin{multicols}{2} % --- break9 --- Interpretation at flow path end
    
    At the end of the flow path (\(z = h, z' = 1\)), this simplifies to:
    \vspace{10mm}
    \columnbreak
    \begin{equation}
        N_D = \frac{C'_h - 1}{1-f}
    \end{equation}
    
\end{multicols}

\bsk

\begin{multicols}{2} % --- break10 --- Explain residence time
    
    The residence time of water along the flow path is:
    \vspace{10mm}
    \columnbreak
    \begin{equation}
        T_f = \frac{h}{\omega}
    \end{equation}
    
\end{multicols}

\begin{multicols}{2} % --- break11 --- Final set of equations
    
    At the end of the flow path:
    \vspace{10mm}


    \columnbreak
    \begin{equation}
        \frac{R_n h}{\phi C_o \omega} = \frac{C'_h - 1}{1-f}
    \end{equation}
    \begin{equation}
        \frac{R_n T_f}{\phi C_o} = \frac{C'_h - 1}{1-f}
    \end{equation}
    
\end{multicols}

\bsk

\begin{multicols}{2} % --- break12 --- Solve for residence time and reaction rate
    
    Solving for residence time \(T_f\) and reaction rate \(R_n\):
    \vspace{10mm}
    \columnbreak
    \begin{equation}
        T_f = \frac{(C_h - C_o)\phi}{(1-f)R_n}
    \end{equation}
    \begin{equation}
        R_n = \frac{(C_h - C_o)\phi}{(1-f)T_f}
    \end{equation}
    
\end{multicols}
% --- break13 --- Final notes
    
$C_h$ is the concentration at the end of the flow path, which is taken to be equal to the concentration of each spring $C$, assuming each spring represents a unique flow path. Convert \(T_f\) from \(10^{-9}\) seconds to years for practical use. 

\begin{table}[H]
    \centering
    %\renewcommand{\arraystretch}{1.3} % Adjust row height
    \begin{tabular}{|c|c|c|c|}
        \hline
        \multicolumn{4}{|c|}{\textbf{Fontorbe}} \\  
        \hline
        \textbf{Parameter} & \textbf{Definition} & \textbf{Units} & \textbf{Formula (Value)} \\  
        \hline
        $\phi$ & Porosity & - & - \\
        $\omega$ & Fluid velocity & m/s & - \\
        $h$ & Length of flow path & m & Variable \\
        $C_h$ & Concentration \@ end of flow path & $\mu$mol/L & Variable \\
        $C_0$ & Initial concentration & $\mu$mol/L & Rain Conc \\
        $f$ & Fraction reprecipitated & - & Order 0.5 \\
        $N_D$ & Non-dimensional number & - & $N_D = \frac{R_n h}{\phi C_0 \omega}$ \\
        $T_f$ & Residence time & $10^{-9}$ s & $T_f = \frac{h}{\omega}$ \\
        $R_n$ & Reaction rate & mol/m$^3$/s & $k\cdot S \cdot \rho \cdot 1000 \cdot X  $ \\
        $k$ & Reaction rate constant & mol/m$^2$/s & - \\
        $S$ & Specific surface area & m$^2$/g & - \\
        $\rho$ & Mineral density & kg/m$^3$ &  \\
        $X$ & Volume fraction of mineral in rock & $g_{min}/g_{rock}$ & 0.2 \\
        \hline
    \end{tabular}
    \caption{Key parameters and definitions for the Fontorbe model.}
    \label{tab:parameters1}
\end{table}
    
}

\end{tcolorbox}


\FloatBarrier

\newpage
\begin{tcolorbox}[
    colback=white,
    colframe=white,
    sharp corners,
    boxrule=0pt,
    breakable,
    width=\dimexpr\textwidth+1cm\relax,
    enlarge left by=-0.5cm,
    leftrule=0mm, rightrule=0mm, toprule=0mm, bottomrule=0mm
]


\subsection{Maher Model Derivation}
\vspace{-5mm}
{\footnotesize

\noindent\rule{\textwidth}{0.5pt}

This model describes a reaction-based approach to solute transport, following Maher and Chamberlain (2013, 2014).

\bsk

The first-order differential equation governing transport and reaction is:\\

\begin{equation}
\frac{dc}{dt} = -\frac{q}{\theta} \frac{dc}{dx} + \sum_{i} \mu_i R_{d,i} \left( 1 - \left( \frac{c}{c_{\text{eq}}} \right)^{n_i} \right)^{m_i} - \sum_{i} \mu_i R_{p,i} \left( 1 - \left( \frac{c}{c_{\text{eq}}} \right)^{n_i} \right)^{m_i}
\end{equation}\\


Where c is the concentration, q is the fluid flux, theta is the volumetric water content, x is the position along the flow path, mu is the stoichiomeric coefficient, R is the rate of reaction for dissolution and precipitation respectively, c$_eq$ is the equilibrium concentration, and n and m are non-linear parameters (Maher and Chamberlain, 2013). 

\begin{equation}
\frac{dc}{dx} =  \frac{q}{\theta}\sum_{i} \mu_i R_{d,i} \left( 1 - \left( \frac{c}{c_{\text{eq}}} \right)^{n_i} \right)^{m_i} - \frac{q}{\phi}\sum_{i} \mu_i R_{p,i} \left( 1 - \left( \frac{c}{c_{\text{eq}}} \right)^{n_i} \right)^{m_i}
\end{equation}\\

\begin{multicols}{2}

Defining the net reaction rate:

\columnbreak

\begin{equation}
R_n = \sum_{i} \mu_i R_{d,i} - \sum_{i} \mu_i R_{p,i}
\end{equation}

\end{multicols}

\begin{multicols}{2}

Maher and Chamberlain describe that the reaction rate decreases linearly with approach to equilibrium.

\columnbreak
\begin{equation}
\frac{dc}{dt} = R_n \left( 1 - \frac{c}{c_{\text{eq}}} \right)
\end{equation}

\end{multicols}

Solving for \( c(x) \), following Maher and Chamberlain (2013):

\begin{equation}
c(x) = c_0 \exp\left(-\frac{R_n \theta x}{q c_{\text{eq}}} \right) + c_{\text{eq}} \left( 1 - \exp\left(-\frac{R_n \theta x}{q c_{\text{eq}}} \right) \right)
\end{equation}\\


\begin{multicols}{2}

We also define the residence time \( T_f \):

\columnbreak

\begin{equation}
T_f = \frac{L\phi}{q}
\end{equation}

\end{multicols}

Thus, at residence time \( T_f \):

\begin{equation}
c(T_f) = c_0 \exp\left(-\frac{R_n T_f}{c_{\text{eq}}} \right) + c_{\text{eq}} \left( 1 - \exp\left(-\frac{R_n T_f}{c_{\text{eq}}} \right) \right)
\end{equation}


\begin{multicols}{2}

Following Maher and Chamberlain (2014), this can be rewritten as:
\columnbreak
\begin{equation}
    C =  \frac{C_{\text{0}}}{1 + \tau D_w / q} + C_{\text{eq}} \frac{\tau D_w / q}{1 + \tau D_w / q}
\end{equation}
    
\end{multicols}

\begin{multicols}{2}

where:

\columnbreak
\begin{equation}
\tau = e^2; \quad D_w = \frac{L\theta R_n}{C_{\text{eq}}}
\end{equation}

\end{multicols}

\begin{multicols}{2}

Rewriting the equation:

\columnbreak

\begin{equation}
    C = \frac{C_{\text{0}} + C_{\text{eq}} \cdot T_f\left(e^2 \cdot R_n / C_{\text{eq}}\right)}{1 + T_f\left(e^2 \cdot R_n / C_{\text{eq}}\right)}
\end{equation}

\end{multicols}

\begin{multicols}{2}

Solving for residence time \( T_f \), and reaction rate \( R_n \) which can be compared to the Fontorbe model:
\columnbreak
\begin{equation}
    T_f = \frac{C_{eq} \cdot \left(C - C_0\right)}{e^2 R_n \left( C_{\text{eq}} - C \right)}
\end{equation}
\begin{equation}
    R_n = \frac{C_{eq} \cdot \left(C - C_0\right)}{e^2 T_f \left( C_{\text{eq}} - C \right)}
\end{equation}

\end{multicols}

Note: Convert \( T_f \) from \( 10^{-3} \)s to years for practical use.\\

\begin{table}[H]
    \centering
    %\renewcommand{\arraystretch}{1.3} % Adjust row height
    \begin{tabular}{|c|c|c|c|}
        \hline  % DOUBLE BOLD LINE
        \multicolumn{4}{|c|}{\textbf{Maher Model Parameters}} \\  
        \hline
        \textbf{Parameter} & \textbf{Definition} & \textbf{Units} & \textbf{Formula (Value)} \\  
        $L$ & Length of flow path & m & Variable \\
        $q$ & Flow rate & m/s & Variable \\
        $\phi$ & Porosity & - & 0.3 (but variable) \\
        $\theta$ & Volumetric water content & - & Variable \\
        $R_n$ & Net reaction rate & mol/L/s & $\rho_{sf} \cdot k \cdot A \cdot X_r $ \\
        $\rho_{sf}$ & Mass mineral / Fluid Volume ratio & g/L & $1000 \cdot \rho_b / \phi$ \\
        $\rho_b$ & Plagioclase density & g/cm$^3$ & - \\
        $k$ & Reaction rate constant & mol/m$^2$/s & - \\
        $A$ & Specific surface area & m$^2$/g & 0.1-1 \\
        $X_r$ & Mineral concentration in fresh rock & $g_{min}/g_{rock}$& Wt\% in rock \\
        $\tau$ & Scaling factor & - & $\tau = e^2$ \\
        $D_w$ & Damkohler Coefficient & m$^2$/s & $D_w = \frac{L \phi R_n}{C_{\text{eq}}}$ \\
        $T_f$ & Residence time & $10^{-6}$ s & $T_f = \frac{L \phi}{q}$ \\
        $C_{eq}$ & Equilibrium concentration & $\mu$mol/L & Max Catchment \\
        $C_0$ & Initial concentration & $\mu$mol/L & Rain Conc \\
        \hline
    \end{tabular}
    \caption{Key parameters and definitions for the Maher model.}
    \label{tab:parameters2}
\end{table}

\FloatBarrier

}

\end{tcolorbox}



% \section{Appendix 2: Dimensional Analysis of \(T_f\) CHANGE THIS!!}

% \subsection*{Fontorbe Model}

% The residence time \(T_f\) is given by:

% \begin{equation}
%     T_f  = \frac{\left(C_h - C_0\right)\cdot\phi}{\left(1-f\right)\cdot R_n}
% \end{equation}

% where the parameters have the following units:

% \begin{itemize}
%     \item \( C_h, C_0 \) (Concentration): \( \mu\text{mol}/\text{L} \)
%     \[
%     1~\mu\text{mol}/\text{L} = \frac{10^{-6}~\text{mol}}{10^{-3}~\text{m}^3} = 10^{-3}~\frac{\text{mol}}{\text{m}^3}
%     \]
%     \item \( \phi \) (Porosity) is dimensionless.
%     \item \( f \) (Fraction reprecipitated) is dimensionless.
%     \item \( R_n \) (Reaction rate) is given by:
%     \begin{equation}
%         R_n = k \cdot S \cdot \rho \cdot 10^3 \cdot X \cdot (1-\phi)
%     \end{equation}
%     where:
%     \begin{itemize}
%         \item \( k \) (Reaction rate constant): \( \left[\frac{\text{mol}}{\text{m}^2 \cdot \text{s}}\right] \)
%         \item \( S \) (Specific surface area): \( \left[\frac{\text{m}^2}{\text{g}}\right] \)
%         \item \( \rho \) (Mineral density): \( \left[\frac{\text{kg}}{\text{m}^3}\right] \) \\
%         Since \( 1~\text{kg} = 10^3~\text{g} \), we write:
%         \[
%         \rho = \frac{10^3~\text{g}}{\text{m}^3}
%         \]
%         \item \( X \) (Volume fraction of mineral in rock) is dimensionless.
%         \item \( (1 - \phi) \) is dimensionless.
%     \end{itemize}
% \end{itemize}

% \subsubsection*{Step-by-Step Dimensional Analysis of \(R_n\)}

% First, multiplying \( k \) and \( S \):

% \[
% k \cdot S = \left(\frac{\text{mol}}{\text{m}^2 \cdot \text{s}}\right) \times \left(\frac{\text{m}^2}{\text{g}}\right) = \frac{\text{mol}}{\text{g} \cdot \text{s}}
% \]

% Next, multiplying by \( \rho \):

% \[
% \left(\frac{\text{mol}}{\text{g} \cdot \text{s}}\right) \times \left(\frac{10^3 \text{g}}{\text{m}^3}\right) = \frac{10^3 \text{mol}}{\text{m}^3 \cdot \text{s}}
% \]

% Since \( X \) and \( (1 - \phi) \) are dimensionless, they do not affect the units.

% Thus, the final unit of \( R_n \) is:

% \[
% R_n = \frac{10^3 \text{mol}}{\text{m}^3 \cdot \text{s}}
% \]

% \subsubsection*{Final Dimensional Analysis of \(T_f\)}

% Substituting the units:

% \[
% T_f = \frac{(C_h - C_0) \cdot \phi}{(1 - f) \cdot R_n}
% \]

% Since \( C_h - C_0 \) has units of:

% \[
% 10^{-3} \frac{\text{mol}}{\text{m}^3}
% \]

% and \( R_n \) has units of:

% \[
% \frac{10^3 \text{mol}}{\text{m}^3 \cdot \text{s}},
% \]

% we get:

% \[
% T_f = \frac{10^{-3} \frac{\text{mol}}{\text{m}^3}}{10^3 \frac{\text{mol}}{\text{m}^3 \cdot \text{s}}}
% \]

% Canceling \( \frac{\text{mol}}{\text{m}^3} \):

% \[
% T_f = \frac{10^{-3}}{10^3} \cdot s = 10^{-6} s
% \]

% \subsubsection*{Conclusion}

% The predicted unit of \( T_f \) is:

% \[
% T_f \sim 10^{-6} \text{ s}
% \]

% \newpage


% \subsection*{Maher Model}

% \subsubsection*{Given Equation for \( T_f \)}

% The residence time \( T_f \) is given by:


% \begin{equation}
%     T_f = \frac{C_{eq} \cdot \left(C - C_0\right)}{e^2 R_n \left( C_{\text{eq}} - C \right)}
% \end{equation}\\


% where the parameters have the following units:

% \begin{itemize}
%     \item \( C_0, C, C_{\text{eq}} \) (Concentrations) in \( \mu \)mol/L:
%     \[
%     1~\mu\text{mol}/\text{L} = 10^{-6}~\frac{\text{mol}}{\text{L}}
%     \]
%     Thus,
%     \[
%     C_0, C, C_{\text{eq}} \sim 10^{-6} \frac{\text{mol}}{\text{L}}
%     \]
%     \item \( R_n \) (Net reaction rate) is defined as:
%     \[
%     R_n = \rho_{sf} \cdot k \cdot A \cdot X_r
%     \]
%     where:
%     \begin{itemize}
%         \item \( \rho_{sf} \) (Mass mineral/Fluid Volume ratio) has units:
%         \[
%         \rho_{sf} \sim \frac{\text{g}}{\text{L}}
%         \]
%         \item \( k \) (Reaction rate constant):
%         \[
%         k \sim \frac{\text{mol}}{\text{m}^2 \cdot \text{s}}
%         \]
%         \item \( A \) (Specific surface area):
%         \[
%         A \sim \frac{\text{m}^2}{\text{g}}
%         \]
%         \item \( X_r \) (Mineral concentration in fresh rock) is dimensionless.
%     \end{itemize}
% \end{itemize}

% \subsubsection*{Step-by-Step Dimensional Analysis of \( R_n \)}

% Expanding \( R_n \):

% \[
% R_n = \left(\frac{\text{g}}{\text{L}}\right) \times \left(\frac{\text{mol}}{\text{m}^2 \cdot \text{s}}\right) \times \left(\frac{\text{m}^2}{\text{g}}\right)
% \]

% Canceling \( \text{g} \) and \( \text{m}^2 \):

% \[
% R_n = \frac{\text{mol}}{\text{L} \cdot \text{s}}
% \]

% Thus, the reaction rate \( R_n \) has the units:

% \[
% R_n \sim \frac{\text{mol}}{\text{L} \cdot \text{s}}
% \]

% \subsubsection*{Dimensional Analysis of \( T_f \)}

% Substituting the units into:


% \begin{equation}
%     T_f = \frac{C_{eq} \cdot \left(C - C_0\right)}{e^2 R_n \left( C_{\text{eq}} - C \right)}
% \end{equation}\\


% - **Numerator:**  
%   \[
%    C_{eq} \cdot \left(C - C_0\right) \sim 10^{-12} \frac{\text{mol}^2}{\text{L}^2}
%   \]

% - **Denominator:**  
%   \[
%   e^2 R_n \left( C_{\text{eq}} - C \right) 
%   \]

%   Since \( e^2 \) is **dimensionless**, we are left with:

%   \[
%   e^2 R_n \sim 10^{-6} \frac{\text{mol}^2}{\text{L}^2 \cdot \text{s}}
%   \]

% Now, dividing:

% \[
% T_f = \frac{10^{-12} \frac{\text{mol}^2}{\text{L}^2}}{\frac{10^{-6}\text{mol}^2}{\text{L}^2 \cdot \text{s}}}
% \]

% Canceling \( \frac{\text{mol}}{\text{L}} \):

% \[
% T_f = 10^{-6} s
% \]

% \subsubsection*{Conclusion}

% The predicted unit of \( T_f \) is:

% \[
% T_f \sim 10^{-6} \text{ s}
% \]

% which corresponds to:

% \[
% T_f = 1~\mu\text{s} \quad (\text{microseconds}).
% \]


