



% \section*{Appendix 1: Strontium Isotopes Mass Balance Mixing Model}
% \addcontentsline{toc}{section}{Appendix 1: Strontium Isotopes Mass Balance Mixing Model}

% Following Faure (2001 - Origin of Igneous Rocks), the Sr concentrations of a two-component mixture can be expressed as:\\

% \begin{equation}
%     [Sr]_M = [Sr]_A f_A + [Sr]_B(1-f_A)
% \end{equation}\\

% These are related to the isotopic composition with:\\

% \begin{equation}
%     \left(\frac{^{87}Sr}{^{86}Sr}\right)_M = \left(\frac{^{87}Sr}{^{86}Sr}\right)_A f_A \frac{[Sr]_A}{[Sr]_M} + \left(\frac{^{87}Sr}{^{86}Sr}\right)_B(1-f_A)\frac{[Sr]_B}{[Sr]_M} 
% \end{equation}\\

% These equations can be combined, yielding:\\

% \begin{equation}
%     \left(\frac{^{87}Sr}{^{86}Sr}\right)_M = \frac{a}{[Sr]_M} + b; \quad \text{with:}
% \end{equation}\\
% \begin{equation}
%     a = \frac{[Sr]_A [Sr]_B \left[ \left( \ddfrac{^{87}Sr}{^{86}Sr} \right)_B - \left( \ddfrac{^{87}Sr}{^{86}Sr} \right)_A \right]}{[Sr]_A - [Sr]_B}
% \end{equation}\\
% \begin{equation}
%     b = \frac{[Sr]_A \left( \ddfrac{^{87}Sr}{^{86}Sr} \right)_A - [Sr]_B \left( \ddfrac{^{87}Sr}{^{86}Sr} \right)_B}{[Sr]_A - [Sr]_B}
% \end{equation}\\

% Plots of $\ddfrac{^{87}Sr}{^{86}Sr}$ against $\ddfrac{1}{Sr}$ that yield straight lines are therefore indicative of mixing trends. 

% \newpage


\section*{Appendix 1: Rain and Hydrothermal Correction Equations}
\addcontentsline{toc}{section}{Appendix 1: Rain and Hydrothermal Correction Equations}
\phantomsection
\label{app:rain_hydrothermal}


Spring water is corrected for rain input according to the average concentration for the closest rain sample collected in this field season. To remove the contribution of the rain the following formula is used for any element X:

\begin{equation}
    [X]_{rain-corrected}  = [X]_{river} - (Cl_{river} - Cl^*_{river})\frac{[X]_{rain}}{[Cl]_{rain}}
\end{equation}
\begin{equation}
    Cl^*_{river} = Cl_{river} - Cl_{rain}; \quad \text{if} \quad Cl_{river} > Cl_{rain}
\end{equation}\\
    

Where $[Cl]^*_{river}$ is is calculated by subtracting the concentration of chloride in the rain from that in the river (Tipper et al, 2006). $Cl^{*}$ is taken to be zero if the concentration of chloride in the rain is greater than concentration of river. Evapotranspiration is not considered by this model, and studies which show that it plays a minor role, accounting for less than 10\% of the hydrological budget in the Himalayas (Andermann et al (2012); Bookhagen and Burbank, 2010). In those cases where $Cl^*$ is not zero then, a primary rain correction is simply:

\begin{equation}
[X]_{rain-corrected}  = [X]_{river} - [X]_{rain}
\end{equation}\\

Once the samples have been corrected for rain input, the remaining $[Cl]^{-}$ is assumed to be derived from hydrothermal springs encountered in the flow path. This is likely to be the case in one of the southern traverses (Traverse 2) in Melamchi which display high chloride concentrations even after the cyclic correction. Hence, the sample with the highest $[Cl]^{-}$ is used to correct:

\begin{equation}
[X]_{hydrothermal-corrected}  = [X]_{rain-corrected} - \frac{[X]}{[Cl]}_{highest-Cl} * [Cl]_{rain-corrected}
\end{equation}

This ensures that all chloride in the corrected sample is removed. The correction uses ionic ratios from the most concentrated water source, which acts as a proxy for the sediment imparting its signature.  In this way, the correction does not affect samples which do not have high Cl (and hence do not have a large hydrothermal contribution), but does decrease the concentration of ions for those that do.

\newpage


\section*{Appendix 2: Derivation of Reactive Transport Models}
\addcontentsline{toc}{section}{Appendix 2: Derivation of Reactive Transport Models}

\renewcommand{\thetable}{A\arabic{table}}  % Change table numbering format

\begin{tcolorbox}[
    colback=white,
    colframe=white,
    sharp corners,
    boxrule=0pt,
    breakable,
    width=\dimexpr\textwidth+1cm\relax,
    enlarge left by=-0.5cm,
    leftrule=0mm, rightrule=0mm, toprule=0mm, bottomrule=0mm
]




\subsection*{\textcite{fontorbeSiliconIsotopicComposition2013} Model Derivation}
\vspace{-5mm}
{\footnotesize

\noindent\rule{\textwidth}{0.5pt}

This model investigates silicon isotopic composition in the Ganges River, assuming constant reaction rates along flow paths. The model is initially built to predict DSi concentrations, but in this study it is adapted to calculate residence times.

\begin{multicols}{2}

    The first-order differential equation governing transport and reaction is given as:
    \vspace{10mm}
    \columnbreak
    \begin{equation}
    \phi \frac{\partial C}{\partial t} = -\omega \phi \frac{\partial C}{\partial z} + R_n(1-f)
    \end{equation} 
    
\end{multicols}

% \begin{table}[H]
%     \centering
%     {\footnotesize
%     \begin{tabular}{|c|c|c|}
%         \hline
%         \textbf{Variable} & \textbf{Definition} & \textbf{Units} \\
%         \hline
%         \(C\) & Concentration in water & \(\mu\)mol/L \\
%         \(\phi\) & Porosity of the medium & - \\
%         \(\omega\) & Fluid velocity along the flow path & m/s \\
%         \(z\) & Distance along the 1D flow path & m \\
%         \(R_n\) & Rate of reaction (dissolution) & mol/m\(^3\)/s \\
%         \(f\) & Fraction of Si reprecipitated in clay minerals & - \\
%         \hline
%     \end{tabular}
%     }
%     %\caption{Key variables and their definitions.}
%     \label{tab:variables}
% \end{table}  
    
\begin{multicols}{2}

To simplify, we introduce non-dimensional variables:
\vspace{10mm}
\columnbreak % --- break3 --- Move to the next column for transformation equations
\begin{equation}
    C' = \frac{C}{C_o},\quad z' = \frac{z}{h},\quad t' = \frac{t\omega}{h}
    \label{eq:transform}
\end{equation}
    
\end{multicols}

\bsk

\begin{multicols}{2}% --- break4 --- Back to left for explaining transformation
    
    Rewriting Equation \ref{eq:transform} using these new variables:
    \vspace{10mm}
    \columnbreak % --- break5 --- Equation for transformed form
    \begin{equation}
        \frac{\partial C'}{\partial t'} = -\frac{\partial C'}{\partial z'} + N_D(1-f)
    \end{equation}
    
\end{multicols}

\bsk

\begin{multicols}{2} % --- break6 --- Explain Damköhler number
    
    The Damköhler number (\(N_D\)) describes the relative importance of kinetic vs transport-controlled settings \parencite{bethkeGEOCHEMICALBIOGEOCHEMICALREACTION}:
    \vspace{10mm}
    \columnbreak
    \begin{equation}
        N_D = \frac{R_n h}{\phi C_o \omega}
    \end{equation}
    
\end{multicols}

\bsk

\begin{multicols}{2} % --- break7 --- Explain steady-state assumption
    
    Assuming a quasi-stationary state (\(\partial C'/\partial t' = 0\)), we get:
    \vspace{10mm}
    \columnbreak % --- break8 --- Equation for steady-state solution
    \begin{equation}
        C' = 1 + z'N_D(1-f)
    \end{equation}
    
\end{multicols}

\bsk

\begin{multicols}{2} % --- break9 --- Interpretation at flow path end
    
    At the end of the flow path (\(z = h, z' = 1\)), this simplifies to:
    \vspace{10mm}
    \columnbreak
    \begin{equation}
        N_D = \frac{C' - 1}{1-f}
    \end{equation}
    
\end{multicols}

\bsk

\begin{multicols}{2} % --- break10 --- Explain residence time
    
    The residence time of water along the flow path is:
    \vspace{10mm}
    \columnbreak
    \begin{equation}
        T_f = \frac{h}{\omega}
    \end{equation}
    
\end{multicols}

\begin{multicols}{2} % --- break11 --- Final set of equations
    
    At the end of the flow path:
    \vspace{10mm}


    \columnbreak
    \begin{equation}
        \frac{R_n h}{\phi C_o \omega} = \frac{C' - 1}{1-f}
    \end{equation}
    \begin{equation}
        \frac{R_n T_f}{\phi C_o} = \frac{C' - 1}{1-f}
    \end{equation}
    
\end{multicols}

\bsk

\begin{multicols}{2} % --- break12 --- Solve for residence time and reaction rate
    
    Solving for residence time \(T_f\) and reaction rate \(R_n\):
    \vspace{10mm}
    \columnbreak
    \begin{equation}
        T_f = \frac{(C - C_o)\phi}{(1-f)R_n}
    \end{equation}

\end{multicols}
% --- break13 --- Final notes
    
$C$ is the concentration at the end of the flow path, which is taken to be equal to the concentration of each spring, assuming each spring represents a unique flow path. Note that given these units, the calculation gives time in \( 10^{-3} \)s. This is converted to years for practical use.\\

\begin{table}[H]
    \setcounter{table}{0}  % Reset table counter to 0
    \renewcommand{\thetable}{A\arabic{table}}  
    \centering
    %\renewcommand{\arraystretch}{1.3} % Adjust row height
    \begin{tabular}{|c|c|c|c|}
        \hline
        \multicolumn{4}{|c|}{\textbf{Fontorbe}} \\  
        \hline
        \textbf{Parameter} & \textbf{Definition} & \textbf{Units} & \textbf{Formula (Value)} \\  
        \hline
        $\phi$ & Porosity & - & 0.1 $^*$\\
        $\omega$ & Fluid velocity & m/s & Variable \\
        $h$ & Length of flow path & m & Variable \\
        $C$ & Concentration \@ end of flow path & $\mu$mol/L & Variable \\
        $C_0$ & Initial concentration & $\mu$mol/L & Rain Input \\
        $f$ & Fraction reprecipitated & - & 0.5$^*$ \\
        $N_D$ & Damkohler Number & - & $N_D = \frac{R_n h}{\phi C_0 \omega}$ \\
        $T_f$ & Residence time & s & $T_f = \frac{h}{\omega\phi}$ \\
        $R_n$ & Reaction rate & mol/m$^3$/s & $\rho \cdot 10^6 \cdot k \cdot S \cdot X $ \\
        $k$ & Dissolution rate constant & mol/m$^2$/s & 10$^{-15*}$ \\
        $S$ & Specific surface area & m$^2$/g & 0.1$^*$ \\
        $\rho$ & Plagioclase density & g/cm$^3$ & 2.7$^*$ \\
        $X$ & Volume fraction of mineral in rock & $g_{min}/g_{rock}$ & 0.2$^*$ \\
        \hline
    \end{tabular}
    \caption{Key parameters and definitions for the Fontorbe model. $^*$ - Values used for calculation.}
    \label{tab:parameters3}
\end{table}
    
}

\end{tcolorbox}


\FloatBarrier

\newpage
\begin{tcolorbox}[
    colback=white,
    colframe=white,
    sharp corners,
    boxrule=0pt,
    breakable,
    width=\dimexpr\textwidth+1cm\relax,
    enlarge left by=-0.5cm,
    leftrule=0mm, rightrule=0mm, toprule=0mm, bottomrule=0mm
]


\subsection*{\textcite{maherRoleFluidResidence2011} Model Derivation}
\vspace{-5mm}
{\footnotesize

\noindent\rule{\textwidth}{0.5pt}

This model describes a reaction-based approach to solute transport, following \textcite{maherHydrologicRegulationChemical2014}. \textcite{maherRoleFluidResidence2011} suggests a dissolution rate law which decreases linearly to zero at equilibrium. The first-order differential equation governing transport and reaction is:

\begin{equation}
    \frac{dC}{dt} = -\frac{q}{\theta} \frac{dC}{dz} + \sum_i \mu_i R_{d,i} \left( 1 - \left( \frac{C}{C_{eq}} \right)^{n_i} \right)^{m_i} - \sum_i \mu_i R_{p,i} \left( 1 - \left( \frac{C}{C_{eq}} \right)^{n_i} \right)^{m_i}
\end{equation}\\

Where \( C \) is the concentration in $\mu$mol/L, \( q \) is the flow rate in m/s, \( \theta \) is the volumetric water content in m$^3$, \( z \) is the position along the flow path in m, \( \mu \) is the stoichiometric coefficient dependent on the reacting minerals, \( R \) is the rate of reaction for dissolution and precipitation respectively in mol/L/s, \( C_{eq} \) is the equilibrium concentration in $\mu$mol/L, and \( n \) and \( m \) are non-linear parameters \parencite{maherRoleFluidResidence2011}. Defining the net reaction rate for a set of packets $i$:

\begin{equation}
    R_n = \sum_i \mu_i R_{d,i} - \sum_i \mu_i R_{p,i}
\end{equation}

Maher and Chamberlain describe that the reaction rate decreases linearly with approach to equilibrium.

\begin{equation}
    \frac{dC}{dt} = R_n \left( 1 - \frac{C}{C_{eq}} \right)
\end{equation}

Solving for \( C \), following \textcite{maherHydrologicRegulationChemical2014}:

\begin{equation}
    C = C_0 \exp \left( -\frac{R_n \theta h}{q C_{eq}} \right) + C_{eq} \left( 1 - \exp \left( -\frac{R_n \theta h}{q C_{eq}} \right) \right)
\end{equation}

The residence time \( T_f \) is defined as:

\begin{equation}
    T_f = \frac{h \phi}{q}
\end{equation}

Thus, at residence time \( T_f \):

\begin{equation}
c(T_f) = c_0 \exp\left(-\frac{R_n T_f}{c_{\text{eq}}} \right) + c_{\text{eq}} \left( 1 - \exp\left(-\frac{R_n T_f}{c_{\text{eq}}} \right) \right)
\end{equation}\\


\begin{multicols}{2}

Following \textcite{maherHydrologicRegulationChemical2014}, this can be rewritten as:
\columnbreak
\begin{equation}
    C =  \frac{C_{\text{0}}}{1 + \tau D_w / q} + C_{\text{eq}} \frac{\tau D_w / q}{1 + \tau D_w / q}
\end{equation}
    
\end{multicols}

\begin{multicols}{2}

Where:

\columnbreak
\begin{equation}
\tau = e^2; \quad D_w = \frac{h\theta R_n}{C_{\text{eq}}}
\end{equation}

\end{multicols}

\begin{multicols}{2}

Rewriting the equation:

\columnbreak

\begin{equation}
    C = \frac{C_{\text{0}} + C_{\text{eq}} \cdot T_f\left(e^2 \cdot R_n / C_{\text{eq}}\right)}{1 + T_f\left(e^2 \cdot R_n / C_{\text{eq}}\right)}
\end{equation}

\end{multicols}

\begin{multicols}{2}

Solving for residence time \( T_f \):

\columnbreak
\begin{equation}
    T_f = \frac{C_{eq} \cdot \left(C - C_0\right)}{e^2 R_n \left( C_{\text{eq}} - C \right)}
\end{equation}
\

\end{multicols}

Note that given these units, the calculation gives time in \( 10^{-3} \)s. This is converted to years for practical use. Also note the $e^2$ term is used because the Maher model considers all paths as if they approach equilibrium.

\begin{table}[H]
    \centering
    %\renewcommand{\arraystretch}{1.3} % Adjust row height
    \begin{tabular}{|c|c|c|c|}
        \hline  % DOUBLE BOLD LINE
        \multicolumn{4}{|c|}{\textbf{Maher}} \\  
        \hline
        \textbf{Parameter} & \textbf{Definition} & \textbf{Units} & \textbf{Formula (Value)} \\ 
        \hline 
        $\phi$ & Porosity & - & 0.1$^*$ \\
        $h$ & Length of flow path & m & Variable \\
        $q$ & Flow rate & m/s & Variable \\
        $C_{eq}$ & Equilibrium concentration & $\mu$mol/L & Max Catchment \\
        $C_0$ & Initial concentration & $\mu$mol/L & Rain Input \\
        $R_n$ & Net reaction rate & mol/L/s & $\rho \cdot 10^6 \cdot k \cdot S \cdot X $ \\
        $\rho$ & Plagioclase density & g/cm$^3$ & 2.7$^*$ \\
        $k$ & Dissolution rate constant & mol/m$^2$/s & 10$^{-15*}$ \\
        $S$ & Specific surface area & m$^2$/g & 0.1$^*$ \\
        $X$ & Volume fraction of mineral in rock & $g_{min}/g_{rock}$& 0.2$^*$ \\
        $\tau$ & Scaling factor & - & $\tau = e^2$ \\
        $D_w$ & Damkohler Coefficient & m$^2$/s & $D_w = \frac{L \phi R_n}{C_{\text{eq}}}$ \\
        $T_f$ & Residence time & s & $T_f = \frac{h \phi}{q}$ \\
        \hline
    \end{tabular}
    \caption{Key parameters and definitions for the Maher model.  $^*$ - Values used for calculation.}
    \label{tab:parameters4}
\end{table}

\FloatBarrier}




\end{tcolorbox}

